
\chapter{Implementation}

The algorithm has been implemented in X lines of Python code as part of the RetroWrite framework. 
RetroWrite is a binary rewriting tool developed for statically instrumenting programs. 
The authors are able to leverage relocation information present in position independent code to produce assembly files that can be reassembled into binaries.
On top of this the framework provides a rewriting API that allows for flexible and expressive transformations of the binary code \cite{Dinesh2020RetroWriteSI}. 

Implementing the algorithm on top of the RetroWrite frameworks allows for the alignment of instructions in existing binaries. This means that you do not need access to 
the source code ... \textbf{benefits of binary rewriting here}

The RetroWrite framework imposes some restrictions on the binary. The binary 

\begin{itemize}
	\item must be compiled as position independent code 
	\item must be \textit{x86\_64} 
	\item must contain symbols and cannot be stripped
\end{itemize}
\cite{hexhive}

The detection of secret dependent branches is not part of the algorithm or the implementation. The user has to provide the algorithm with the address of the target instruction. 
At the time of writing secret dependent branching instructions need to be identified through manual inspection. However, research has shown that static detection of these side channels is possible, though 
this is currently limited to the MSP430 architecture \cite{MSP430Detection}.

Intel provides some data regarding the latencies of commonly used instructions \cite{intel-ref-manual} but this data is not complete.
To obtain better data Abel et. al developed novel algorithms to infer the latency throughput, and port usage based on automatically-generated microbenchmarks \cite{uops}. 
The authors claim that their results are more accurate and precise than existing work. 
Another source of data on instruction latencies is provided by Agner Fog who provides the results of his own measurements \cite{fog_2021}. 

The data provided by Abel et. al is  used as the primary source of instruction latencies. In the case where an instruction is not covered by their work the data 
provided by Agner Fog and Intel are used as a secondary source. If a program contains an instruction that is not covered by any of the datasets then the program cannot
be aligned. The exception to this rule are branching instructions. There is no latency information available about these instructions in any of the sources. 
To account for this  all branching instructions are aligned with new branching instructions. To preserve the control flow of the program the target of 
the branching instruction is equal to the address of the next instruction. 
