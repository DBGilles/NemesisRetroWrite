

\documentclass[master=cws,masteroption=ai, english]{kulemt}



\setup{% Remove the "%" on the next line when using UTF-8 character encoding
inputenc=utf8,
title={Mitigating Microarchitectural Timing Side Channel Attacks With Binary Instrumentation},
author={Gilles De Borger},
promotor={Prof.\,dr.\, Danny Hughes},
assessor={Sam Michiels \and Kim Wuyts},
assistant={Sam Michiels \and Majid Salehi}}

\setup{font=lm}


\usepackage{amssymb} % for \smallsetminus
\usepackage{pifont}% http://ctan.org/pkg/pifont
\newcommand{\cmark}{\ding{51}}%
\newcommand{\xmark}{\ding{55}}%
\usepackage{listings}
\usepackage{graphicx}
\usepackage{subcaption}
\usepackage{xurl}
\usepackage{url}
\usepackage[ruled,vlined, linesnumbered]{algorithm2e}
\newcommand\mycommfont[1]{\footnotesize\ttfamily\textcolor{blue}{#1}}
\SetCommentSty{mycommfont}



\usepackage{amsmath}

\begin{document}

\begin{preface}
I would like to express my sincere gratitude to everyone who has supported me throughout my degree. 
In particular I want to thank my daily supervisor, Majid Salehi, whose guidance helped me complete my thesis. 
I want to thank my parents, without whom I would have never gotten to where I am today.
Finally I want to thank my sister for providing much needed emotional support.
\end{preface}

\tableofcontents

\begin{abstract}
Protected Module Architectures are a promising line of research to safeguard sensitive applications executing in an untrusted operating system. 
These architectures ensure that an untrusted OS is prevented from accessing the module's code or data. 
Recent research has shown, however, that PMAs are still vulnerable to controlled-channel attacks, a type of side-channel attack that leverages 
the attacker's high level of control over the OS to open additional side-channels. 

One such attack is Nemesis. 
Nemesis exploits the CPU's interrupt mechanism to leak micro-architectural timings from protected modules. 
The attacker is able to infer information about the secret-dependent control flow of a program based on differences in instruction timings 
in branches of conditional jump instructions. 
This thesis proposes a novel algorithm for automatically transforming existing binaries to close these timing leaks. 
Additional instructions are inserted into branches of a conditional jump instruction to ensure that corresponding instructions have identical 
latencies, making the branches indistinguishable to an attacker who is able to observe instruction timings.
The proposed algorithm applies these transformations through the use of binary rewriting. 
Unlike previous solutions that require either recompilation of the source code, or modifications to the hardware, the proposed algorithm can be applied to 
commercial off-the-shelf binaries. This makes it an attractive solution for use in the field. 

An implementation is presented for the Intel x86\_64 architecture.
A number of experiments are performed to evaluate the effectiveness and correctness of the algorithm. 
The results indicate that the proposed solution effectively close all timing leaks without altering the program outcome. 
\end{abstract}

\begin{abstract*}
Als gevolg van de toenemende populariteit van IoT-apparaten en cloud-computingdiensten, wordt software vaak uitgevoerd op platformen van derden. 
IoT apparaten ondersteunen vaak de mogelijkheid extra software te installeren. Grote providers van cloud-computingdiensten, zoals Amazon en Google, bieden de mogelijkheid aan 
om vrijwel eender welk stuk software te installeren op hun servers. 
Het resultaat hiervan is dat er meer en meer aandacht besteed wordt aan het beveiligen van software die uitgevoerd wordt op een onbetrouwbaar besturingssysteem. 
Een veelbelovend onderzoeksveld binnen deze context is het gebruik van \textit{Protected Module Architectures}. 
Deze zijn architecturen die hardwarematig de isolatie van verschillende modules in het systeem garanderen, en die ervoor zorgen dan een onbetrouwbaar besturingssysteem geen toegang
heeft tot de code of gegevens van de software. 

Recent onderzoek heeft echter aangetoond dat het nog steeds mogelijk is gevoelige informatie te extraheren uit programma's binnen dergelijke architecturen door middel van \textit{controlled-channel attacks}. 
Deze zijn een type van \textit{side-channel attack} waarbij het besturingssysteem onder controle van aan aanvaller staat, en die gebruik maken van de verhoogde graad van controle over het besturingssysteem om nieuwe \textit{side-channels} te open. 
\textit{Side-channel attacks} trachten gevoelige informatie te halen uit software door het meten van fysieke parameters zoals voedingsstroom, en uitvoeringstijd. 
\textit{Controlled-channel attacks} maken gebruik van \textit{system events} (e.g. \textit{page faults}, \textit{cache flushes}, en \textit{interrupts}) om nieuwe 
\textit{side-channels} te openen. 

Een voorbeeld van zo een \textit{controlled-channel attack} is Nemesis. Nemesis misbruikt het interruptmechanisme om micro-architecturale instructietimings te bepalen voor
instructies van applicaties binnen een \textit{Protected Module Architecture}. 
Indien er conditionele sprongen zijn waarbij overeenkomstige instructies verschillende instructietimings hebben is het mogelijk om op basis van deze metingen te bepalen welke van de twee takken uitgevoerd wordt
Door deze instructietimings te verzamelen voor verschillende inputs en de verschillen te vergelijken
is het mogelijk om te bepalen welke paden van de \textit{control flow} het programma heeft gevolgd voor een gegeven input. 
Aan de hand hiervan kan de aanvaller inferenties maken over geheime data die gebruikt wordt om \textit{control flow} beslissingen te maken. 

Deze thesis stelt een nieuwe oplossing voor die een automatische transformatie toepast op programma's om dergelijke \textit{timing side-channels} te sluiten. 
Dit algoritme voegt aan de hand van \textit{binary rewriting} extra instructies toe om verschillen in instructietimings tussen takken van conditionele sprongen te elimineren. 
Als gevolg is het niet meer mogelijk om het onderscheid te maken tussen takken van een conditionele sprong op basis van de gemeten instructietimings.
\textit{Binary rewriting} is het aanpassen van een gecompileerd programma zonder gebruikt te maken van de originele code. 
Omdat deze oplossing niet berust op het aanpassen van de hardware, of op het hercompileren van de code, is het mogelijk om deze toe te passen op bestaande 'off-the-shelf' programma's. 

Het voorgestelde algoritme manipuleert het programma door aanpassingen uit te voeren op de \textit{Control Flow Graph} (CFG). 
Dit is een datastructuur die de \textit{control flow} van een programma voorstelt, bestaande uit knopen en zijden. Hierbij bevat elke knoop een verzameling van instructies die telkens 
als een geheel uitgevoerd wordt. De zijden geven aan hoe de \textit{control flow} van het programma springt tussen de verschillende noden. 
De CFG moet voldoen aan twee eigenschappen vooraleer de nodige instructies toegevoegd kunnen worden. 
Ten eerste moeten alle paden naar een gegeven node dezelfde lengte hebben, en ten tweede 
moeten de lengtes van alle paden naar een blad in de CFG even lang zijn. Een blad is een node met geen uitgaande zijden. 
De eerste fase van het algoritme vult de CFG met extra knopen aan zodat de graaf voldoet aan bepaalde eigenschappen. 
De tweede fase voegt dan niveau-gewijs instructies toe aan de knopen van de CFG om zo de \textit{timing leaks} te sluiten. 

Het algoritme is ge\"implementeerd voor de Intel x86\_64 architectuur aan de hand van RetroWrite, een \textit{binary rewriting} framework. 
Een aantal experimenten zijn uitgevoerd op een verzameling van testprogramma's om te bepalen of het algoritme doeltreffend is in het sluiten van de \textit{timing leaks}, of het dit doet 
zonder de output van het programma te veranderen, en om de impact op de performantie van het programma te evalueren. 
Uit de resultaten blijkt dat het algoritme correct alle \textit{timing leaks} kan sluiten. Het doet dit ook zonder de uitkomst te veranderen bij alle programma's, met \'e\'en uitzondering. 
Deze uitzondering bevat een functie die opgeroepen wordt in \'e\'en tak van een conditionele sprong, waarbij de functie ook nog de globale staat van het programma aanpast. Het algoritme kopieert deze oproep naar de andere tak, waardoor
het de globale staat van het programma gewijzigd wordt waar dit voorheen niet gebeurde. 

Het algoritme voorgesteld in deze thesis is doeltreffend in het sluiten van \textit{timing leaks}. 
In tegenstelling tot eerdere oplossing voor het sluiten van \textit{timing leaks} kan het toegepast worden op bestaande, 'off-the-shelf' binaire bestanden, omdat het gebruikt maakt van \textit{binary rewriting}.  
Verder kan het algoritme ge\"implementeerd worden voor een architectuur indien er hiervoor een \textit{binary rewriting} framework bestaat. 
\end{abstract*}


\listoffiguresandtables


\mainmatter 


\lstdefinelanguage
   [x64]{Assembler}     % add a "x64" dialect of Assembler
   [x86masm]{Assembler} % based on the "x86masm" dialect
   % with these extra keywords:
   {morekeywords={CDQE,CQO,CMPSQ,CMPXCHG16B,JRCXZ,LODSQ,MOVSXD, %
                  POPFQ,PUSHFQ,SCASQ,STOSQ,IRETQ,RDTSCP,SWAPGS, %
                  rax,rdx,rcx,rbx,rsi,rdi,rsp,rbp, %
                  r8,r8d,r8w,r8b,r9,r9d,r9w,r9b, %
                  r10,r10d,r10w,r10b,r11,r11d,r11w,r11b, %
                  r12,r12d,r12w,r12b,r13,r13d,r13w,r13b, %
                  r14,r14d,r14w,r14b,r15,r15d,r15w,r15b}} % etc.

\chapter{Introduction}
\label{cha:introduction}

Because of the increasing popularity of IoT devices more and more embedded computing devices are being connected to the Internet. 
These devices are often more susceptible to being exploited because they support software extensibility. 
Additionally because these devices are connected to a network the risk increases since attacks can be done remotely. 
An important technique for securing such devices is the use of hardware support for virtual memory and processor privilege levels.
The OS can build on this support to isolate a process from any other malicious processes on the device. 
However, this introduces a sizable software layer however which is difficult to get sufficiently secure \cite{psma}.
If the attacker controls the OS then its capabilities for attacking a process on the devices increase considerably \cite{citation needed}. 

Maene et. al state that \textit{the goal of trusted computing is to develop technologies which give users guarantees about the behaviour of the software running on their devices}.
An important aspect of trusted computing is therefore to protect software even when attackers have full control of the system. 
One means of achieving this is through the use of Protected Module Architectures (PMAs). 
These architectures seperate critical components into protected modules, also called enclaves,
that are isolated from one another through hardware \cite{trusted-computing-architectures}.  

A number of Protected Module Architectures (PMAs) have been developed to address this problem, both by researchers and industry. 
% volgende zin trekt mogelijks te veel op zin in nemesis paper
PMAs have been developed for both low-end microcontrollers found in embedded systems \cite{trustlite, smart} as well as high-end processors \cite{isox}.
One architecture developed for embedded systems is Sancus. Sancus is a security architecture that can provide strong isolation guarantees on networked embedded systems, 
and has been implemented on a modified TI MSP430 micro-controller \cite{sancus}. 
At the higher end of the spectrum there is Intel SGX. 
Intel SGX is an extension  added to the Intel architecture that allows applications to instantiate enclaves. 
Enclaves are areas in the application's memory that are protected from access from outside of the enclave, even from 
privileged software such as the OS \cite{SGX}. 

Research has shown that it is possible to extract information from protected applications in PMAs. In their work Xu et. al introduce a novel type of side-channel attack 
called controlled-attacks. These attacks are categorized by untrusted operating systems that create side-channels through its extensive control of the system.
The authors were able to leverage the OS's high degree of control over the system to extract large amounts of data from applications which were until 
then safe from side-channel attacks \cite{xu}. 

More recently Van Bulck et. al \cite{Nemesis} developed Nemesis, a controlled-channel attack that leverages the interrupt mechanism to extract sensitive information from 
enclaved applications. The authors were able to exploit timing differences in the latency between the arrival of an interrupt request (IRQ) and the execution of the first instruction in the 
interrupt service routine (ISR). They state that their attack is \textit{based on the key observation that an IRQ during a multi-cycle instruction increases the interrupt 
latency with the number of cycles left to execute}. By carefully and deliberately interrupting a process at the right time the authors were able to infer the duration of the interrupted instruction. 
Potential attackers can use this information to determine where the instruction is situated in the program's control flow. When the instruction is part of a secret-dependent branch the 
attacker is able to infer some information about the secret, successfully leaking sensitive information from the program. Van Bulck et. al \cite{Nemesis} showed that this attack is applicable to 
the whole computing spectrum. They were able to apply their attack to the aforementioned Sancus architecture as well as Intel SGX enclaves.  

Figure \ref{fig:pseudo-assembly} illustrates how such an attack might work with a piece of assembly pseudocode. An attacker who is in control of the OS could carefully interrupt the program right
after the conditional jump at line 5. Depending on the value of register r1 the next interrupted instruction is either the addition instruction at line 4 or the multiplication instruction at line 7. 
By measuring the interrupt latency the attacker can infer which of the two instructions was being executed at the time of the interrupt and, more importantly, infer if the value in register r0 is equal to 0. 


\lstset{language=[x64]Assembler, numbers=left, stepnumber=1, frame=single}
\begin{figure}

    \begin{lstlisting}
	CMP r1, $0
	JEQ .l1
	.l1: 
	ADD r1, r2 		; 1 cycle instruction
	JMP .end
	.l2: 
	MUL r1, r2 		; 2 cycle instruction
	JMP .end
	\end{lstlisting}
	\caption{Assembly pseudo-code with a secret-dependent branch that is vulnerable to Nemesis attack}
	\label{fig:pseudo-assembly}
\end{figure}

This paper presents a novel algorithm for automatically transforming a program in order to remove any timing leaks. It achieves this by addressing the core cause of the vulnerability: differences in 
latencies between corresponding instructions in secret-dependent branches. Corresponding instructions are instructions that are the same distance away from a branching instruction. 
The proposed algorithm inserts additional instructions such that all corresponding instructions have the same latency without changing affecting the program output. 

The main contributions of this paper are:
\begin{enumerate}
\item The paper presents a novel algorithm for automatically transforming a program to remove any timing leaks.
\item The paper presents an implementation of this algorithm for the Intel x86-64 architecture. 
\item The paper presents an evaluation of the algorithm based on a suite of benchmark programs. 
\end{enumerate}


\chapter{Background}
\label{cha:background}
This chapter aims to provide additional information on some concepts related to the thesis. 
Section \ref{sec:pma} describes protected module architectures and the motivation behind them. 
Section \ref{sec:nemesis} outlines the workings of the Nemesis attack, providing an example to further illustrate it. 
Finally section \ref{sec:rewriting} introduces the concept of binary rewriting and discusses its uses. 
\section{Protected Module Architectures}
\label{sec:pma}
Because of the increasing popularity of IoT devices, more and more embedded computing devices are being connected to the Internet. 
These devices are often more susceptible to being exploited because they support software extensibility. 
Additionally, because these devices are connected to a network, the risk increases since attacks can be done remotely. 

An important technique for securing such devices is hardware-supported virtual memory and processor privilege levels.
The OS can build on this support to isolate a process from any other malicious processes on the device. 
However, this introduces a sizable software layer however which is difficult to get sufficiently secure \cite{psma}.
If the attacker controls the OS then its capabilities for attacking a process on the devices increase considerably. 

Maene et al. \cite{trusted-computing-architectures} state that \textit{the goal of trusted computing is to develop technologies which give users guarantees about the behaviour of the software running on their devices}.
An important aspect of trusted computing is therefore to protect software even when attackers have full control of the system. 
One means of achieving this is through the use of Protected Module Architectures (PMAs). 
These architectures seperate critical components into protected modules, also called enclaves,
that are isolated from one another through hardware.  

A number of Protected Module Architectures (PMA) have been developed to address this problem, both by researchers and industry. 
PMAs have been developed for both low-end microcontrollers found in embedded systems \cite{trustlite, smart} as well as high-end processors \cite{isox}.
One architecture developed for embedded systems is Sancus. 
Sancus is a security architecture that can provide strong isolation guarantees on networked embedded systems, 
and has been implemented on a modified TI MSP430 micro-controller \cite{sancus}. 
At the higher end of the spectrum, there is Intel SGX.
Intel SGX is an extension  added to the Intel architecture that allows applications to instantiate enclaves. 
Enclaves are areas in the application's memory that are protected from access from outside of the enclave, even from 
privileged software such as the OS \cite{SGX}. 

Research has shown that it is still possible to extract information from protected applications in PMAs. In their work Xu et al. \cite{Xu} introduce a novel type of side-channel attack 
called controlled-attacks. These attacks are categorized by untrusted operating systems that create side-channels through its extensive control of the system.
The authors were able to leverage the OS' high degree of control over the system to attack applications that were previously out of reach of side-channel attacks, and were able to 
extract large amounts of data in a single run. 

\section{Nemesis Side-Channel Attack}
\label{sec:nemesis}
More recently, Van Bulck et al. \cite{Nemesis} developed Nemesis, a controlled-channel attack that leverages the interrupt mechanism to extract sensitive information from 
enclaved applications. The authors were able to exploit timing differences in the latency between the arrival of an interrupt request (IRQ) and the execution of the first instruction in the 
interrupt service routine (ISR). They state that their attack is \textit{based on the key observation that an IRQ during a multi-cycle instruction increases the interrupt 
latency with the number of cycles left to execute}. By carefully and deliberately interrupting a process at the right time, the authors were able to infer the duration of the interrupted instruction. 
Potential attackers can use this information to determine where the instruction is situated in the program's control flow. When the instruction is part of a secret-dependent branch, the 
attacker is able to infer some information about the secret, successfully leaking sensitive information from the program. Van Bulck et al. \cite{Nemesis} showed that this attack is applicable to 
the whole computing spectrum. They were able to apply their attack to the aforementioned Sancus architecture as well as Intel SGX enclaves.  

Figure \ref{fig:pseudo-assembly} illustrates how such an attack might work with a piece of assembly pseudocode. An attacker who is in control of the OS could carefully interrupt the program right
after the conditional jump at line 5. Depending on the value of register r1, the next interrupted instruction is either the addition instruction at line 4, or the multiplication instruction at line 7. 
By measuring the interrupt latency the attacker can infer which of the two instructions was being executed at the time of the interrupt, and, more importantly, infer if the value in register r0 is equal to 0. 



\lstset{language=[x64]Assembler, numbers=left, stepnumber=1, frame=single}
\begin{figure}

    \begin{lstlisting}
	CMP r1, $0
	JEQ .l1
	.l1: 
	ADD r1, r2 		; 1 cycle instruction
	JMP .end
	.l2: 
	MUL r1, r2 		; 2 cycle instruction
	JMP .end
	\end{lstlisting}
	\caption{Assembly pseudo-code with a secret-dependent branch that is vulnerable to Nemesis attack}
	\label{fig:pseudo-assembly}
\end{figure}


\section{Binary Rewriting}
\label{sec:rewriting}
Binary rewriting is the alteration of a compiled program without having the source code at hand. 
Applications of binary rewriting include observing programs during execution, optimizing programs using run-time patching, and 
hardening applications against attacks. In the case of dynamic binary rewriting the rewriting happens during execution of the program. 
Static binary rewriting, on the other hand, occurs before the binary is executed \cite{rewriting-survey}. 
Binary rewriting tools have been developed for both low-end architectures found in embedded devices \cite{microsbs} as well as high-end architectures found in home computers and servers 
\cite{ instruction-punning, Dinesh2020RetroWriteSI, E9Patch}. 


%\chapter{Design}
\label{cha:design}

\section{Introduction}
This section outlines the approach to mitigate Nemesis style attacks. The cause of the vulnerability exposed by Nemesis-style attacks are differences in 
latency traces. Attackers are able to look at the differences in these latency traces and infer which secret-dependent branches were taken by the program, 
leaking information from the program. The algorithm outlined in this section aims to prevent this by aligning the latency traces along various paths by 
inserting additional instructions into corresponding nodes of two different branches, ensuring that there are no differences that can leak information. 

The proposed algorithm performs a number of operations on a programs Control Flow Graph (CFG). The two main operations are the insertion of additional nodes into the graph and the alignment of a set of nodes. 
Because not all CFG structures are suitable for alignment the first stage of the algorithm inserts nodes into the graph to ensure alignment is possible. The second stage consists of alignment corresponding nodes 
in the graph. 

Section \ref{sec:property} will define the property that needs to hold for a program in order for Nemesis-style attacks to be mitigated. Section \ref{sec:cfg} introduces the CFG data structure and translates the aforementioned property to such CFG structures. Finally, sections \ref{seq:equalising} and \ref{seq:alignment} describe the insertion and alignment of nodes, respectively. 

\section{Nemesis-sensitive property}\label{sec:property}
In their paper Pouyanrad et. al have formally defined the Nemesis-Sensitive property.  Let $region^{then}(ep)$ and $region^{else}(ep)$ capture the set of execution points belonging to the branch target and the other region of some branching instruction $ep$. Let $ep^i$ be the i'th instruction in a region. A program P with a secret-dependence branch in $ep$ and $region^{then}(ep)$ 
and region $region^{else}(ep)$ with the same number of execution points, satisfies the nemesis-sensitive property if and only if:  

\begin{equation} \label{eq:nemesisProperty}
    \begin{split}
    \forall ep^i \in region^{then}(ep) : \forall ep^j \in region^{else}(ep) \text{ \textit{such that} } i=j :  \\ 
    (s_{ep^i} \xrightarrow[]{t} s_{ep^i_{next}}) \land (s_{ep^j} \xrightarrow[]{t'} s_{ep^j_{next}}) \iff t = t' \\ 
    \end{split}
\end{equation}
\cite{MSP430Detection}
The relation $s \xrightarrow[]{t} s'$ models the transition between program states $s$ and $s'$, declaring that the transition between $s$ and $s'$ takes a time $t$. 
For a given instruction this time $t$ is equal the instruction's latency. This property states that for any two corresponding instructions in the branches their latencies should be the same.

If this nemesis-sensitive property holds for a program then an attacker is not able to infer which branch was taken by the program based on instruction latencies.

\section{CFG}\label{sec:cfg}
The Control Flow Graph (CFG) is a data structure that represents the control flow of a program. A CFG consists of nodes $V$ and directed edges $E$. Each node $V$ contains a 
contiguous sequence of instructions. Any branching instruction can only occur at the end of such a sequence, and an instruction that is the target of 
of a branching sequence can only occur at the start. 
An edge is drawn from node $v$ to node $v'$ if and only if the last instruction in $v$ can be followed by the first instruction in $v'$ 
when following program control flow. The algorithm only considers branching instructions that are binary in nature, so a node in the CFG can have at most 2 successors. 
By construction of this data structure a branching instruction will always be the last instruction in a node. A node is said to be secret-dependent if its last instructions is a 
secret-dependent branching instruction. 

Each node has a latency sequence associated with it, equal to the latencies of the node's instructions.  
A latency trace along a path of the CFG is then equal to the concatenation of the latency sequences of each node along the path.
Figure \ref{fig:exampleCFG} shows an example of a such a CFG, along with the original program it is created from. The CFG also contains the latency for each instruction. Note that by convention the only node with no incoming edges is considered the starting node of the CFG. 

\begin{figure}
\centering
\begin{subfigure}{.4\textwidth}
  \centering
  
    \begin{lstlisting}[language=C]
int main(){
        int a = 10; 
        int b = 20; 
        if (a < b){
            int temp = b; 
            b = a; 
            a = temp; 
        } 
        return a;  
}\end{lstlisting}
  \caption{C program}
  \label{fig:c_program}
\end{subfigure}%
\begin{subfigure}{.7\textwidth}
  \centering
  \includegraphics[width=.7\textwidth]{images/sample_program_graph.png}
  \caption{Corresponding CFG}
  \label{fig:c_program_cfg}
\end{subfigure}
\caption{Example program with corresponding CFG}
\label{fig:exampleCFG}
\end{figure}

Following the property described in section \ref{eq:nemesisProperty}, the nemesis-sensitive property can be defined for a node in the CFG. Let $v$ be a secret-dependent node. 
Let $v_f$ be a node such that all paths from $v$ to some leaf go through $v_f$. Then $region^{then}(v)$ can be defined as the set of nodes reachable following the 
first of $v$'s outgoing edges up to and including $v_f$ and $region^{else}(v)$ as the set of nodes reachable following the other outgoing edge up to and including $v_f$. Any differences in latencies between two nodes that are descendants of $v_f$ cannot be used to infer information about the secret dependent node $v$. All nodes below $v_f$ therefore do not have to be considered. If no such node $v_f$ exists then the regions simply consists of all nodes reachable from $v$ through one of its outgoing edges. 

Let $n^i \in region(v)$ be a node such that there is a path going to it from node $v$ of length $i$. The depth of $region(v)$ is defined as being the length of the longest path from $v$ to some node $v' \in region(v)$ that does not contain a cycle.

A secret-dependent node $v$ and $region^{then}(v)$ and $region^{else}(v)$ with the same depth satisfies the nemesis-sensitive property if and only if 
\begin{equation} \label{eq:nemesisPropertyNode}
    \begin{split}
    \forall n^i \in region^{then}(v) : \forall n^j \in region^{else}(v) \text{ \textit{such that} } i=j :  \\ 
    latencies(n^i) = latencies(n^j)
    \end{split}
\end{equation}
where $latencies(n)$ is a function mapping a node $n$ to its latency sequence. 

Figure \ref{fig:regionExamples} illustrates how the borders of each region is defined. The secret-dependent node is marked in green, while the two branches are marked in red and blue. In the second example, the node marked in purple belongs to both regions. In example \ref{fig:regionExampleA} there is node node such that all paths from the secret-dependent node to a leaf go through it, so the regions extend all the way to the leaves. In example \ref{fig:regionExampleB} all paths that start in the secret-dependent node go through the node 6. Any differences in nodes 7 and 8 can only be used to infer information about the branch in node 6. These nodes therefore do not have to be considered.  


\begin{figure}
 \centering
 \subfloat[]{\includegraphics[height = 6cm]{images/nemesis-property-example-1.png}\label{fig:regionExampleA}}
 \subfloat[]{\includegraphics[height = 6cm]{images/nemesis-property-example-2.png}\label{fig:regionExampleB}} 

 \captionof{figure}{then-else regions for secret-dependent nodes}
 \label{fig:regionExamples}
\end{figure}

\section{Equalising}\label{seq:equalising}
There are 2 structures commonly found in a program's control flow graph that make it impossible to enforce the nemesis-sensitive property for a node as defined in the previous section, 
shown in figure \ref{fig:problemStructures}.  Before aligning the nodes an equalising step is applied to ensure these structures are do not occur in the CFG. 

\subsection{Problematic structure}
The first such structure occurs when the program contains some sequence of instructions that is only executed if some condition is true.
In this structure there will be some node that has two paths to it of different lengths. One path will contain the node that corresponds to the conditional instructions, while the other path will 
not contain this node. Because the paths have some overlapping nodes and because the paths have different lengths it is impossible to equalize their latency traces by inserting additional instructions, since inserting instructions in one of the paths will also insert instructions into the other path. If these paths start at a secret-dependent node it is therefore impossible to 
ensure that the nemesis-sensitive property holds. 

The second problematic structure occurs when one of the branches is shorter than the other one, as shown in figure \label{fig:unequal}. In such cases there will be some nodes in 
one branch that have no corresponding nodes in the other branch, making it impossible to align them. 

The nemesis-sensitive property as defined in section \ref{eq:nemesisPropertyNode} entails that it is impossible for a node to satisfy the property if one of these 
structures occurs in its branches, since in both cases the regions have different depths. The first stage of the algorithm therefore consists of first equalizing all path lengths and then 
equalizing branches. Algorithms \ref{alg:equalizePaths} and \ref{alg:equalizeBranches} depict pseudo-code for equalizing paths lengths and equalizing branches respectively. 

\subsection{Equalize paths}
To equalize all path lengths starting from some secret-dependent node $v$, first a subset of the graph's nodes are extracted such that 
only the regions $region^{then}(v)$ and $region^{else}(v)$ are considered. This is done by applying a modified breadth-first search that stops early when the 
current node dominates all of the leaves reachable from $n$. By definition all paths between $n$ and a leaf will pass through this node, in which case the 
any of the following nodes no dot have to be considered. Next the length of the longest path is computed  from $v$ to all nodes in the sub-graph. 
If there exists edges such that the length of the longest path to the tail and the length of the longest path to the head differ by more than one then 
there are at least 2 paths to the head of different lengths. In this case additional nodes are inserted between the tail and the head to equalize these path lengths. 

\subsection{Equalize branches}
The branches of a CFG can be equalized in a similar way. Given some secret-dependent node $v$ a subset of the graph's nodes are extracted. 
Next the lengths of the longest paths are computed for all leaves of the sub-graph, as well as the length of the longest path from the root to some leaf. 
If for some leaf the longest path to it is too short then additional nodes are inserted as predecessors to this leaf. 

\begin{figure}
 \centering
 \subfloat[optional node]{\includegraphics[height = 6cm]{images/optional.png}\label{fig:optional}}
 \subfloat[unequal branches]{\includegraphics[height = 6cm]{images/unequal.png}\label{fig:unequal}} 

 \captionof{figure}{problematic structures in CFG}
 \label{fig:problemStructures}
\end{figure}

\section{Alignment} \label{seq:alignment}

During the alignment stage the nodes of the CFG are aligned in a level-wise manner. During this stage additional instructions are inserted into nodes to ensure that corresponding instructions have the same latency.
Algorithm \ref{alg:align} depicts pseudocode for this stage of the algorithm. 

The level of a node is defined as being the distance between the root of the graph and the node. After the first stage of the algorithm all paths to a given nodes have the same length. This makes the level of a node a well 
defined value. The alignment stage iterates over all the levels of the sub-graph and aligns the nodes found at that level. 

\subsection{Basic Operation}
The core of the alignment step consists of repeatedly selecting a reference instruction from one of the nodes and inserting instructions into the other nodes such that the latencies match. The node from which the 
reference instruction is selected is called the reference node. The reference node can change throughout the algorithm. 

Because an instruction is potentially added to each node that is not the reference node, the reference node needs to have at least as many instructions as the node with the largest number of instructions. This ensure that at some point all nodes have the same number of instructions. Let $n_{max}$ be the number of instructions in the longest node. The set of candidate nodes then consists of all nodes that have $n_{max}$ instructions. The reference 
node is then selected from this set of candidates. 

An index variable is used to keep track of the position of the reference instruction. This variable is initially zero and is incremented every iteration. Any instructions that have an index smaller than this variable 
are balanced. The reference instruction is selected by first selecting the reference node and then selecting the instruction in this node that has an index equal to the index variable. 

Given a reference instruction, the algorithm iterates over all nodes that are not the reference node and verifies if the corresponding instruction has the same latency. 
If the two latencies are not equal, or if the node has no corresponding instruction,  then a new instruction is inserted into the node at the index equal to the index variable. 
Once this has been done for all nodes then all instructions with an index smaller than or
equal to the index variable will be balanced and the index variable can be incremented. 

\subsection{Selecting the Reference Node}
If there are multiple candidates nodes then the algorithm selects one of the nodes where the corresponding reference instruction is not a branching instruction or a return statement. From this set a reference node 
is arbitrarily selected. If there are no such nodes then any node can be selected as the reference node.  

\subsection{Constructing NOP instruction}
For each latency class a template NOP instruction has been determined. The instruction can be inserted into the program as-is if it has no effect on the program state, i.e. it does not modify any register values. 
If the instruction does modify some register the algorithm selects a registers that can safely be used. This needs to be a register that is not in use at the time of execution of the instruction. 

The function is statically analyzed to determine which registers are free to use for this purpose. There are two types of free registers. 
A register can be free because its current value is no longer used, i.e. because it is overwritten at some later point without
being read first. Alternatively a register can be free because it isn't used anywhere in the current function. In the latter case, however, it is possible that the register is in use by the caller, 
since there is no guarantee that the caller stored all the registers it uses. 

If a register of the first type exists then it can be used as the operand of the NOP instruction and the resulting instruction 
can be inserted as-is into the node. If no such registers exists, a free register of the second type is selected, and additional instructions are inserted into the program to ensure that the 
original value of the register is not lost. In the root of the CFG additional instructions are inserted to push the register value onto the stack, while in every leaf instructions are inserted 
that pop the value from the stack. 

If there are no free registers available, any register is arbitrarily selected. Additional instructions are inserted before and after the NOP instruction to push and pop the register value. To ensure the nodes are 
still balanced these push and pop instructions are inserted across all nodes of the current level. 

\subsection{branching instructions}
In the cases where the reference instruction is a branching instruction, the newly inserted instruction will also be a branching instruction. The target of the branching instruction is the successor of the node where 
the node is inserted. 


\begin{algorithm*}
  \SetAlgoLined
  \DontPrintSemicolon
  
  \SetKwFunction{equalizePathsFnName}{EqualizePathLengths}
  \SetKwFunction{extractSubgraphFNName}{ExtractSubGraph}
  \SetKwFunction{computeLongestPathFNName}{ComputeLongestPathLengths}
  \SetKwFunction{CreateNodeFNName}{CreateNode}
  \SetKwFunction{InsertNodeFNName}{InsertNodeBetween}
  \SetKwFunction{edges}{Edges}
  \SetKwFunction{nodes}{Nodes}
  \SetKwFunction{leaves}{Leaves}
  \SetKwFunction{topologicalorder}{TopologicalOrder}
  \SetKwFunction{successors}{Successors}
  \SetKwFunction{predecessors}{Predecessors}
  \SetKwFunction{addNode}{AddNode}
  \SetKwFunction{addEdge}{AddEdge}
  \SetKwFunction{removeEdge}{RemoveEdge}
  \SetKwFunction{max}{Max}
  \SetKwProg{equalizePaths}{Procedure}{}{}
  
   \equalizePaths{\equalizePathsFnName{g: CFG, v: Node}}{
   \textit{subgraph} $\leftarrow$ \extractSubgraphFNName(g, v) \\  
   \textit{longestPathLengths} $\leftarrow$ \computeLongestPathFNName{subgraph, v} \\
   \ForAll{(u, v) $\in$ \edges{subgraph}}{
   diff $\leftarrow$  \textit{longestPathLengths}[u] - longestPathLengths[v] \\
   \uIf{diff $>$ 1}{
        \textit{head} $\leftarrow$ v \\ 
        \For{$i \in$ 1, 2, ..., diff-1}{
%            \FSetLatencies(node, target\_latencies)
            \textit{newNode} $\leftarrow$ \CreateNodeFNName{} \\
            \InsertNodeFNName{newNode, u, head} \\
            \textit{head} $\leftarrow$ newNode \\
        }
   }
   }
   }{}
   
     \SetKwProg{ComputeLongestPathLengths}{Function}{}{}
  \ComputeLongestPathLengths{\computeLongestPathFNName{g: CFG, s: Node}}{
  dist = \{n : -1 $\vert$  n $\in$ \nodes{g} \} \\
  dist[s] $\leftarrow$  0 \\
  \ForAll{n $\in$ \topologicalorder{g}}{
    
    \ForAll{succ $\in$ \successors{n}}{
        dist[succ] $\leftarrow$ \max{dist[succ], dist[n] + 1} \\
    }
    
  }
  \KwRet dist \\
  }
  
   \SetKwFunction{computeImmediateDominators}{computeImmediateDominators}
  \SetKwProg{ExtractSubGraph}{Function}{}{}
  \ExtractSubGraph{\extractSubgraphFNName{g: CFG, n: Node}}{
  \textit{immedidateDominators} $\leftarrow$  \computeImmediateDominators{g, n} \\
  \textit{leafDominators} $\leftarrow$  \{ $u$ $\in$ \nodes{g} $\vert(u, v) \in$ \textit{leafDominators} $ \land v \in$ \leaves{g} \} \\
  
  \uIf{$\vert$ leafDominators $\vert$ = 1}{
    dominator $\leftarrow$  leafDominators[0] \\
  } \uElse{
    dominator $\leftarrow$  $\varnothing$ \\
  }
  
 
  \textit{subgraphNodes} $\leftarrow$  [n] \\
  \textit{adjacentNodes}  $\leftarrow$  \successors{n} \\
  \While{$\vert$ adjacentNodes $\vert$ > 0}{
    currentNode $\leftarrow$  adjacentNodes[0] \\
    adjacentNodes $\leftarrow$  adacentNodes $ \setminus $ \{currentNode\} \\
    \uIf{currentNode = dominator}{
        \KwRet \textit{subgraphNodes}
        }
    \textit{adjacentNodes} $\leftarrow$  \textit{adjacentNodes}  $ \bigcup $ $( \successors{currentNode} \setminus \textit{adjacentNodes} )$ \\
  } 
  \KwRet \textit{subgraphNodes}
  }
  
  \caption{Equalize Path Lengths}
  \label{alg:equalizePaths}
\end{algorithm*}

\begin{algorithm*}
    \SetAlgoLined
    \DontPrintSemicolon
    \SetKwProg{equalizeBranches}{Procedure}{}{}
    \SetKwFunction{equalizeBranchesFNName}{EqualizeBranches}

    \equalizeBranches{\equalizeBranchesFNName{g: CFG, n: Node}}{
    \textit{subgraph} $\leftarrow$ \extractSubgraphFNName(g, v) \\  
    \textit{longestPathLengths} $\leftarrow$ \computeLongestPathFNName{subgraph, v} \\
    \textit{maxPathLength} $\leftarrow$ \max{$\{ longestPathLengths[v] \, \vert \,  v \in  \leaves{subgraph}\}$} \\
    \ForAll{leaf $\in$ \leaves{subgraph}}{
        diff $\leftarrow$ \textit{longestPathLengths[leaf]} - maxPathLength \\
        
        \uIf{diff $>$ 0}{
                $v$ $\leftarrow$ leaf \\
                \For{i $\in$ 1, 2, ..., diff}{
                    newNode $\leftarrow$ \CreateNodeFNName{} \\
                    \addNode{g, newNode} \\
                    \ForAll{$p \in$ \predecessors{v}}{
                        \addEdge{g, (p, newNode)} \\
                        \removeEdge{g, (p, v)} \\
                    }
                    \addEdge{g, (newNode, v)} \\
                }
        }
    }
    }
    \caption{Equalize Branches}
    \label{alg:equalizeBranches}
\end{algorithm*}

\begin{algorithm*}      
    \SetAlgoLined
    \DontPrintSemicolon
    
    \SetKwProg{alignCFG}{Procedure}{}{}
    \SetKwProg{alignNodes}{Procedure}{}{}
    
    \SetKwFunction{alignCFGFNName}{AlignCFG}
    \SetKwFunction{alignNodesFNName}{AlignNodes}
    \SetKwFunction{computeDistanceFromNode}{ComputeDistanceFromNode}
    
    \alignCFG{\alignCFGFNName{g: CFG, n: Node}}{
    subgraph $\leftarrow$ \extractSubgraphFNName{g, v} \\
    pathLengths $\leftarrow$ \computeDistanceFromNode{subgraph, v} \\
    levels $\leftarrow$ \{ l $\; \vert \; u \in$ \nodes{subgraph} $\land$ pathLengths[u] = l \} \\
    \ForAll{l $\in$ levels}{
        levelNodes $\leftarrow$ \{ $u \; \vert \; u \in$ \nodes{subgraph} $\land$ pathLengths[u] = l \} \\
        \alignNodesFNName{subraph, levelNodes} \\
    }
    }
    
    \SetKwFunction{countInstruction}{CountInstructions}
    \SetKwFunction{selectReferenceNode}{SelectReferenceNode}
    \SetKwFunction{getNodeInstruction}{GetNodeInstruction}
    \SetKwFunction{latency}{Latency}
    \SetKwFunction{isBranch}{IsBranch}
    \SetKwFunction{isReturn}{IsReturn}
    
    \SetKwFunction{selectRegister}{SelectRegister}
    \SetKwFunction{getNOPInstruction}{GetNOPInstruction}
    \SetKwFunction{getBranchInstruction}{GetBranchInstruction}
    \SetKwFunction{insertInstruction}{insertInstruction}
    
    \alignNodes{\alignNodesFNName{g: CFG, ns : NodeSet}}{
    index $\leftarrow$ 0 \\ 
    \While{True}{
        nodeLengths $\leftarrow$ \{ node: \countInstruction{node} $\; \vert \;$ node $\in$ \nodes{g} \} \\
        candidates $\leftarrow$ \{n $\; \vert \;  n \in$ \nodes{g} $\land$  nodeLengths[n] = \max{nodeLengths} \} \\
        referenceNode $\leftarrow$ \selectReferenceNode{candidates} \\
        referenceInstruction $\leftarrow$ \getNodeInstruction{referenceNode, index} \\
        
        \ForAll{node $\in$ \{n $\vert$ n $\in$ \nodes{g} $\land$ n $\neq$ referenceNode \}}{
            \uIf{index $<$ nodeLength[node] $\; \land \;$  \latency{\getNodeInstruction{node, index}} = \latency{referenceInstruction}}{
                continue
            }
            \uIf{\isBranch{referenceInstruction}}{
                newInstruction $\leftarrow$ \getBranchInstruction{}\\ 
                \insertInstruction{node, newInstruction} \\
            } \uElse{
                reg $\leftarrow$ \selectRegister{} \\ 
                newInstruction $\leftarrow$ \getNOPInstruction{\latency{referenceInstruction}, reg} \\ 
                \insertInstruction{node, newInstruction} \\
            }
        }    
    }
    }
    
    \SetKwProg{selectReferenceNode}{Function}{}{}
    \SetKwFunction{selectReferenceNodeFNName}{SelectReferenceNode}

    \selectReferenceNode{\selectReferenceNodeFNName{candidates: NodeSet, index: Integer}}{
        \For{n $\in$ candidates}{
            candidateInstruction $\leftarrow$ \getNodeInstruction{n, index} \\
            \uIf{$\lnot$ (\isBranch{candidateInstruction} $\lor$ \isReturn{candidateInstruction}) }{
                \KwRet n \\
            }
        }
        \KwRet candidates[0] \\
    }
    
    

    
    \caption{Align CFG}
    \label{alg:align}
\end{algorithm*}


\chapter{Implementation}

\subsubsection{RetroWrite}
The algorithm has been implemented in X lines of Python code as part of the RetroWrite framework. 
RetroWrite is a binary rewriting tool developed for statically instrumenting C and C++ binaries. 
The authors are able to leverage relocation information present in position independent code to produce assembly files that can be reassembled into binaries.
On top of this the framework provides a rewriting API that allows for flexible and expressive transformations of the reconstructed assembly files \cite{Dinesh2020RetroWriteSI}

% volgende zin bijna letterlijk uit RetroWrite paper 
The RetroWrite frameworks implements a logical abstraction for rewriting passes to operate on.
These come in the form of data structures that represent the logical units of a program,
Each of these data structure provide an interfaces for analyzing and modifying the underlying data. 
One such logical abstraction is the \textit{InstructionWrapper}. 
This datastructure stores among other things, the instruction address, the mnemonic, and the operand string, and provides an interface
for modifying the underlying instruction and for prepending or appending additional instructions.  
The \textit{Function} datastructure contains a set of instructions and a function that maps each instruction to all instruction that can follow it. 

The proposed algorithm is implemented as an additional abstraction layer on top of these data structures. 
Each node of the CFG consists of a sequence of \textit{InstructionWrapper}s. 
The edges of the CFG are reconstructed based on data inside the \textit{Function} instances. 
The CFG data structure implements  an interface for the insertion of additional nodes into the graph and the insertion of additional instructions into nodes. 
This interface is built on top of the RetroWrite API, so all modifications to the CFG result in modification to the underlying \textit{InstructionWrapper} instances. 
RetroWrite provides functionality for writing the instruction to assembly files. This assembly file can then be compiled using any of-the-shelf compiler. 


The RetroWrite framework imposes some restrictions on the binary. The binary must be compiled as position independent code, it must contain instructions from x86\_64 architecture, and 
it cannot be stripped of symbols \cite{hexhive}. 
As a result the implementation only supports binaries that meet these restrictions. 

\subsubsection{Binary Rewriting}
% TODO dit meer uitgebreid 
Implementing the algorithm as part of a binary rewriting framework has two main advantages. 
Firstly the algorithm can be applied to existing binaries. 
Secondly the algorithm does not require the source code of the program. It can be applied to third-party software. 


\subsubsection{Secret-dependent branches}
The detection of secret dependent branches is not part of the algorithm or the implementation. The user has to provide the algorithm with the address of the target instruction. 
At the time of writing secret dependent branching instructions need to be identified through manual inspection. However, research has shown that static detection of these side channels is possible, though 
this is currently limited to the MSP430 architecture \cite{MSP430Detection}.

\subsubsection{Instruction Latencies}
The construction of NOP instructions as described in section \ref{sec:nop} is based on data that measures the latency of instructions in the x86-64 architecture.
Intel provides some data regarding the latencies of commonly used instructions \cite{intel-ref-manual} but this data is not complete.
To obtain better data Abel et. al  \cite{uops} developed novel algorithms to infer the latency throughput, and port usage based on automatically-generated microbenchmarks. 
The authors claim that their results are more accurate and precise than existing work. 
Another source of data on instruction latencies is provided by Agner Fog who provides the results of his own measurements \cite{fog_2021}. 

The data provided by Abel et. al is  used as the primary source of instruction latencies. In the case where an instruction is not covered by their work the data 
provided by Agner Fog and Intel are used as a secondary source. If a program contains an instruction that is not covered by any of the datasets then the program cannot
be aligned. The exception to this rule are branching instructions. There is no latency information available about these instructions in any of the sources. 
To account for this  all branching instructions are aligned with new branching instructions. To preserve the control flow of the program the target of 
the branching instruction is equal to the address of the next instruction. 


\chapter{Evaluation}
\label{cha:evaluation}
The proposed algorithm is evaluated by running a number of experiments on a benchmark suite of programs. 
Section \ref{sec:benchmark-suite} outlines how this benchmark suite was constructed. 
The setup of the experiments is then described in section \ref{sec:setup}. Finally section \ref{sec:results} discusses the results. 

\section{Benchmark Suite}
\label{sec:benchmark-suite}
Winderix et al. \cite{WinderixHans} have created the first benchmark suite of programs with timing side-channel vulnerabilities. This suite consists of a 
collection of synthetic programs with a wide range of control-flow patterns, as well as third party benchmark programs from different sources. 
To evaluate the proposed algorithm a subset of this benchmark suite was selected. 

All programs that contain loops inside vulnerable branches were discarded from the synthetic programs in the benchmark suite
since they are not supported by the proposed algorithm. 
The original authors of the Nemesis attack provide two case studies to demonstrate their attack. 
The first case study is a password comparison routine from the Texas Instruments MSP430 Bootstrap Loader (BSL). 
The second case study is secure keypad application that guarantees secrecy of its PIN code \cite{Nemesis}.
Both of these are included in the benchmark suite created by Winderix et al.\cite{WinderixHans}, and are also selected as a benchmark for the proposed algorithm. 

Nemesis and the benchmark suite created by Winderix et al. \cite{WinderixHans} are implemented for the Sancus environment. Any pieces of code specific to this
environment have been removed from the benchmark programs. The semantics of the programs remain unchanged. 

One additional synthetic programs was added to the benchmark suite to evaluate a case that was not yet covered. This program contains a call to a function that 
modifies a non-local variable though a pointer. This function is only called in one branch of a secret-dependent branch. 

\section{Experiment Setup}
\label{sec:setup}
The algorithm is evaluated using three metrics. The first metric aims to measure the effectiveness of the algorithm. A static analysis tool was developed to verify 
whether or not program satisfies the Nemesis-sensitive property as specified in section \ref{sec:goals}.
Given a program and a set of secret-dependent branches, this tool partitions instructions into sets according to their positions in secret dependent branches. Following the notation of section \ref{sec:goals}, let \textit{ep} be a secret dependent branch, and let $ep^n$ be the n'th instruction in a region, then define the set 
\begin{equation} \label{eq:toolSets}
    ep_i = \{ ep^n |i = n \land  (ep^n \in region_{then}(ep) \lor ep^n \in region_{else}(ep)\}
\end{equation}
The static analysis verifies that both the regions have the same number of execution points, and that for each set $ep_i$ it holds that all 
instruction have the same latency.

The second metric aims to measure the correctness of the algorithm. The algorithm is considered to work correctly if it does not change the program output. For each program in the benchmark suite a number of input values were determined such that all possible paths of the program control flow were covered.
These values were supplied as inputs to both the original program and the balanced program, generating two output values. The output values were then compared to 
verify that the algorithm correctly modified the program without changing the output. 

The effect on the program's performance is evaluated by measuring the increase in the sum of the latencies along paths in the programs CFG. 
To measure this increase, CFGs are constructed from the original binary and from the modified binary. 
For each path in the original CFG its corresponding path in the modified CFG is determined. To do so a mapping is created that maps all nodes in the original CFG to their corresponding node in the balanced CFG. 
This mapping takes into account the condition of a branching instruction, and can be defined inductively. 
The root of the original CFG is mapped to the root of the root of the balanced CFG. 
If two nodes are mapped and they both have one successor, then their successors are mapped. 
If two nodes are mapped and they have two successors, then then nodes that are reached if the branching condition is true are mapped,
and those that are reached if the condition is false are mapped.

Formally, let $G$ denote the original CFG, and let $G'$ denote the modified CFG. Let $succ(n)$ be the successors of node $n$, and let $succT(n)$ be 
the successor of node N when the branching condition is true. Let $F$ be the function that maps between the two CFGs.  
\begin{enumerate}
    \item $\begin{aligned}[t]
    F(root(G)) = root(G')
\end{aligned}$
\item $\begin{aligned}[t]
    F(n) = n' \land succ(n) = \{s\} \land succ(n')=\{s'\} \\ 
    \implies F(s)=s'
\end{aligned}$
\item $\begin{aligned}[t]
    F(n) = n' \land succ(n) = \{s, t\} \land succ(n')=\{s', t'\} \land succ_T(n)=s \land succ_T(n') = s'\\
    \implies F(s)=s', F(t) = t'
\end{aligned}$
\end{enumerate}
Let $p$ be a path in $G$
$$ p: p_1 \rightarrow p_2 \rightarrow ... \rightarrow p_n$$
Then its corresonding path in $G'$ is defined as follows 
$$ p': F(p_1) \rightarrow F(p_2) \rightarrow ... \rightarrow F(p_n)$$

This definition requires that $G$ and $G'$ are isomorphic. If during the first stage of the algorithm additional nodes were inserted in the CFG
then this will not be true. Therefore, before being able to evaluate the effect on runtime, the first stage of the algorithm has to be reapplied on $G$ such 
that it is isomorphic to $G'$

To evaluate the effect on runtime performance, the sum of the latencies along all relevants paths in G are compared to the sum of the latencies of their corresponding paths. A relevant path is a path that starts in secret-depdendent node and ends in a final node of one of the branches. Any nodes that do not belong to such a path 
are not affected by the algorithm, and are therefore not considered in this evaluation. 

    
\section{Results}
\label{sec:results}
The results of the experiments are summarized in figure \ref{fig:experiment results}.
A check mark in the second column indicates that all timing-leaks have been removed. 
A check mark in the third column indicates that the program output has not changed. 
The third column contains the increase in the sum of the latency along various paths in the program, expressed as a percentage increase.

The algorithm was able to ensure the Nemesis-sensitive property holds for all programs, as verified by the static analysis tool described in the previous section. This effectively closes the 
timing leaks found in the program. 

In all but one test case the algorithm had no effect on the program output. 
The erroneous test case contains a call to a function that modifies the global state of the program in one of its secret dependent branches. 
During balancing of the program this function call is copied to the other branch. 
Because the function call has side effects the final output of the program is different. 

The effect on performances ranges from an increase by a factor of 1.8, to no increase at all. 
There are two factors that cause larger increases in the sum of the latencies along a path. The first is the difference in depths between two secret dependent branches. 
If one branch is significantly deeper then a larger number of nodes will be inserted in the other branch, lengthening the paths found in the shorter branch. 
Because each inserted node has to be aligned, a larger number of instructions will be inserted. 

This effect is the cause for the relatively large increase in latencies in the benchmark  program \textit{multifork}. 
This program contains a secret-dependent switch statement with three cases, which is equivalent to three nested secret-dependent if statements. 
In the best case the program only has to check the first condition. If it is true, the body is executed and the control flow branches past the switch statement. 
In the worst case the program has to check all three conditions before being able to leave the switch statement. 
The path followed by the control flow in the best case is much shorter than the path followed in the worst case. 
Because all paths need to have the same length, the number of instructions inserted in this shorter path is much larger. 

The second factor is the presence of a larger conditional code block. All instructions found in the conditional node also have to be inserted in the path where the condition is false. 
As the number of instruction in the conditional node increases, so does the sum of the latencies in the other path.

This is reflected in the difference between the experiment results for the benchmarks \textit{triangle} and \textit{fork}. The structure of the CFGs is similar for both these benchmarks as both contain a conditional node.
Benchmark \textit{fork} assign the result of an expression to a variable inside this node, whereas \textit{triangle} simply assigns a constant. The number of instruction in the optional node is therefore larger in \textit{fork}, resulting 
in a larger effect on performance.

\begin{figure}
    \centering
    \begin{tabular}{| l | c | c | c c c c c c |}
    \hline
    Benchmark & Effectiveness & Correctness & \multicolumn{6}{ c|}{Performance} \\ 
     & & & path1 & path2 & path3 & path4 & path5 & path6\\
     \hline 

    call        & \cmark & \cmark & 1.32 & 1.17 & &  & & \\  
    call2       & \cmark & \xmark & 1.25 & 1.22 & &  & & \\
    diamond     & \cmark & \cmark & 1.35 & 1.06 & 1.06 & & &  \\ 
    fork        & \cmark & \cmark & 1.43 & 1.15 & & & &  \\  
    ifcompound  & \cmark & \cmark & 1.31 & 1.26 & 1.11 & 1.11 & 1.09 & 1.09  \\
    indirect    & \cmark & \cmark & 1.44 & 1.32 & 1.20 & 1.11 & & \\ 

    multifork   & \cmark & \cmark & 1.80 & 1.58 & 1.41 & 1.41 & &   \\
    triangle    & \cmark & \cmark & 1.30 & 1.16 & & & &  \\
  	\hline
  	bsl         & \cmark & \cmark & 1.42 & 1.00 & & & &  \\ 
	keypad      & \cmark & \cmark & 1.67 & 1.55 & 1.45 & 1.02 & 1.12 &   \\  
%    sharevalue  & Y & Y & 1.25 & 1.0 & & & &  \\
%	\hline
%	geometric mean & & & 1.67 & 1.55 & 1.45 & 1.02 & 1.12 & \\
	\hline 
    \end{tabular}
    \caption{Results of the evaluation experiments}
    \label{fig:experiment results}
\end{figure}



\chapter{Related Work}
\label{cha:related}
This chapter discusses a number of countermeasures that have been proposed for closing timing side-channels. 
These can broadly be categorized as being either software-based or hardware-based. 
Hardware-based solution are based on modification to the architecture and are discussed in section \ref{sec:hardware}, whereas the software-based solutions discussed in section \ref{sec:software} are implemented at the language level \cite{Barthe}.  
\section{Software-based approaches}
\label{sec:software}

Popular software-based approaches to closing timing-leaks include constant-time policies and the program counter model \cite{Barthe}. 
Constant time policies require that memory access and control-flow should not depend on secret data. 
Unfortunately writing code that adheres to these policies can be difficult, since it requires knowledge of the compiler, and requires developers to deviate from 
conventional programming practices.
A number of solutions have been proposed to verify if a program adheres to constant-time policies \cite{verify-constant-time, Barthe}.

Molnar et al. \cite{programcounter} first introduced the program counter model in their work, proposing methods for the detection and mitigation of control-flow side channel attacks. 
The authors consider the case where an adversary is able to make an observation of a side channel at each step of the computation. 
The result is a sequence of observations $T= (T_1, T_2, ... T_n)$ called a \textit{transcript}.
The \textit{program counter model} or \textit{PC model} is then the model where each value of the transcript is the processor's program counter during the computation. 
A program is then said to be  PC-secure if this transcript is secure. 
The authors state that \textit{any program that is PC-secure will also be secure against timing attacks}.
Based on this definition of PC-security the authors introduce a code transformation for creating PC-secure C code.
The authors note that they made a number of assumptions in their work. 
Although these assumptions do not hold for a number of architectures the authors are confident 
that they are reasonable for some embedded devices. 	

Winderix et al. \cite{WinderixHans} recently proposed a new algorithm that aligns instructions in corresponding branches in a way similar to the algorithm outlined in this text, 
by first equalizing path lengths and then aligning nodes. 
They have implemented their algorithm for the Sancus architecture, in the form of a compiler pass in the LLVM structure. Their solution supports loops nested within secret-dependent branches 
and as a result covers a larger set of programs. Unlike the solution proposed in this work, however, their solution cannot be applied to off-the-shelf binaries, as it requires access to the source code and recompilation.  
 
\section{Hardware-based approaches}
\label{sec:hardware}

An orthogonal approach is to close timing leaks using hardware-based solutions. 
Recently Busi et al. \cite{busi} proposed an approach to extend architectures with non-interruptible enclaved execution such that they can also support interruptions without breaking 
the existing isolation properties. Based on this approach, they proposed  a design for interruptible enclaves that are resistent against interrupt attacks. 
This design is based on an earlier version of Sancus with non-interruptible enclaves. They modify the architecture to add padding cycles whenever the enclave is interrupted, effectively closing timing leaks. 

A limitation of their approach is that their design is based on the assumption that the timing of instruction is predictable. This is generally not the case for more complex architectures such as Intel's x86\_64.
Additionally their approach requires hardware modifications, which means it cannot be applied to off-the-shelf and existing devices. 



\chapter{Conclusion}

\section{Discussion}
One limitation of the proposed algorithm is the lack of support for cycles within secret-dependent branches.
However, the evaluation shows that the algorithm is effective in closing timing leaks in programs where this does not occur. 
This indicates that it is possible modify the algorithm such that it is able to close these leaks even in the presence of cycles. 

The root cause of the issue is the fact that a section of the first branch will be executed a higher number of times than the corresponding section in the second branch. 
As a result one of the latency traces will be longer, even when the relevant nodes are aligned.
A solution that addresses this would have to make more extensive changes to the program. 
Before aligning the nodes, the structure of the cycle would have to be duplicated into the second branch such that the corresponding section is is executed the same number of times. 
This requires additional analysis to determine which register contains the loop counter and how many time it is incremented. 
The duplication of the cycle also requires more significant changes than those implemented by the current algorithm. % hier nog aangeven waarom / welke aanpassingen? 
Due to the added complexity such a solution is not included in the proposed algorithm and is left to future research. 

A second limitation is the incomplete coverage of the latency data. There are certain instructions for which there is no available data. 
If these instructions are encountered in branches of  a secret-dependent branching instruction then the algorithm is not able to close the timing leaks. 
However this issue was not encountered during evaluation of the algorithm. 
This indicates that the most commonly used instructions are present in the data and that the coverage of the data is sufficient to close timing leaks in most programs. 

The evaluation of the effectiveness of the algorithm is based on a statistical analysis of the program before and after alignment. 
If the instruction latencies are fully deterministic then the static analysis tool can correctly predict the actual run-time instruction latencies. The resulting analysis 
is then sufficient for demonstrating that the algorithm correctly closes all timing leaks. 
In the presence of advanced micro-architectural features the instruction latencies are to some extend random, however. 
As a result the run-time latencies diverge from the predicted latencies. 
Although the results indicate that all timing leaks are closed it is possible that some differences still exist between branches because of these random variations. 
Because of this additional experiments are needed to fully verify if the algorithm is effective for complex architectures with non-deterministic latencies.
Empirical measurements in the form of latency traces are needed to determine if an attacker can still distinguish between branches of a secret-dependent conditional node. 



\subsection{Secret-dependent branches}
The detection of secret dependent branches is not part of the algorithm or the implementation. The user has to provide the algorithm with the address of the target instruction. 
At the time of writing secret dependent branching instructions need to be identified through manual inspection. However, research has shown that static detection of these side channels is possible, though 
this is currently limited to the MSP430 architecture \cite{MSP430Detection}.

\section{Future work}


\backmatter 

\bibliographystyle{abbrv}
\bibliography{references}

\end{document}
