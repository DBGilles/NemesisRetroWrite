

\documentclass[master=cws,masteroption=ai, english]{kulemt}



\setup{% Remove the "%" on the next line when using UTF-8 character encoding
inputenc=utf8,
title={Mitigating Microarchitectural Timing Side Channel Attacks With Binary Instrumentation},
author={Gilles De Borger},
promotor={Prof.\,dr.\, Danny Hughes},
assessor={Sam Michiels \and Kim Wuyts},
assistant={Sam Michiels \and Majid Salehi}}

\setup{font=lm}


\usepackage{amssymb} % for \smallsetminus
\usepackage{pifont}% http://ctan.org/pkg/pifont
\newcommand{\cmark}{\ding{51}}%
\newcommand{\xmark}{\ding{55}}%
\usepackage{listings}
\usepackage{graphicx}
\usepackage{subcaption}
\usepackage{xurl}
\usepackage[ruled,vlined, linesnumbered]{algorithm2e}
\newcommand\mycommfont[1]{\footnotesize\ttfamily\textcolor{blue}{#1}}
\SetCommentSty{mycommfont}



\usepackage{amsmath}

\begin{document}

\begin{preface}
I would like to express my sincere gratitude to everyone who has supported me throughout my degree. 
In particular I want to thank my daily supervisor Majid Salehi for offering guidance whenever I needed it. 
I want to thank my parents, without whom I would have never gotten to where I am today.
Finally I want to thank my sister for providing much needed emotional support.
\end{preface}

\tableofcontents

\begin{abstract}
Protected Module Architectures are a promising line of research to safeguard sensitive applications executing in an untrusted operating system. 
These architectures ensure that an untrusted OS is prevented from accessing the module's code or data. 
Recent research has shown, however, that PMAs are still vulnerable to controlled-channel attacks, a type of side-channel attack that leverages 
the attacker's high level of control over the OS to open additional side-channels. 

One such attack is Nemesis. 
Nemesis exploits the CPU's interrupt mechanism to leak micro-architectural timings from protected modules. 
The attacker is able to infer information about the secret-dependent control flow of a program based on differences in instruction timings 
in branches of conditional jump instructions. 
This thesis proposes a novel algorithm for automatically transforming existing binaries to close these timing leaks. 
Additional instructions are inserted into branches of a conditional jump instruction to ensure that corresponding instructions have identical 
latencies, making the branches indistinguishable to an attacker who is able to observe instruction timings.
The proposed algorithm applies these transformations through the use of binary rewriting. 
Unlike previous solutions that require either recompilation of the source code, or modifications to the hardware, the proposed algorithm can be applied to 
commercial off-the-shelf binaries. This makes it an attractive solution for use in the field. 

An implementation is presented for the Intel x86\_64 architecture.
A number of experiments are performed to evaluate the effectiveness and correctness of the algorithm. 
The results indicate that the proposed solution effectively close all timing leaks without altering the program outcome. 
\end{abstract}

\begin{abstract*}
Als gevolg van de toenemende populariteit van IoT-apparaten en cloud-computingdiensten, wordt software vaak uitgevoerd op platformen van derden. 
IoT apparaten ondersteunen vaak de mogelijkheid extra software te installeren. Grote providers van cloud-computingdiensten, zoals Amazon en Google, bieden de mogelijkheid aan 
om vrijwel eender welk stuk software te installeren op hun servers. 
Het resultaat hiervan is dat er meer en meer aandacht besteed wordt aan het beveiligen van software die uitgevoerd wordt op een onbetrouwbaar besturingssyteem. 
Een veelbelovend onderzoeksveld binnen deze context is het gebruik van \textit{Protected Module Architectures}. 
Deze zijn architecturen die hardware-matig de isolatie van verschillende modules in het systeem garanderen, en die ervoor zorgen dan een onbetrouwbaar besturingssyteem geen toegang
heeft tot de code of gegevens van de software. 

Recent onderzoek heeft echter aangetoond dat het nog steeds mogelijk is gevoelige informatie te extraheren uit programma's binnen dergelijke architecturen door middel van \textit{controlled-channel attacks}. 
Deze zijn een type van \textit{side-channel attack} waarbij het besturringsysteem onder controle van aan aanvaller staat, en die gebruik maken van de verhoogde graad van controle over het besturinggsysteem om nieuwe \textit{side-channels} te open. 
\textit{Side-channel attacks} trachten gevoelige informatie te halen uit software door het meten van fysieke parameters zoals voedingsstroom, en uitvoeringstijd. 
\textit{Controlled-channel attacks} maken gebruik van \textit{system events} (e.g. \textit{page faults}, \textit{cache flushes}, en \textit{interrupts}) om nieuwe 
\textit{side-channels} te openen. 

Een voorbeeld van zo een \textit{controlled-channel attack} is Nemesis. Nemesis misbruikt het interruptmechanisme om micro-architecturale instructietimings te bepalen voor
instructies van applicaties binnen een \textit{Protected Module Architecture}. 
Indien er conditionele sprongen zijn waarbij overeenkomstige instructies verschillende instructietimings hebben is het mogelijk om op basis van deze metingen te bepalen welke van de twee takken uitgevoerd wordt
Door deze instructietimings te verzamelen voor verschilende inputs en de verschillen te vergelijken
is het mogelijk om te bepalen welke paden van de \textit{control flow} het programma heeft gevolgd voor een gegeven input. 
Aan de hand hiervan kan de aanvaller inferenties maken over geheime data die gebruikt wordt om \textit{control flow} beslissingen te maken. 

Deze thesis stelt een nieuwe oplossing voor die een automatische transformatie toepast op programma's om dergelijke \textit{timing side-channels} te sluiten. 
Dit algoritme voegt aan de hand van \textit{binary rewriting} extra instructies toe om verschillen in instructietimings tussen takken van conditionele sprongen te elimineren. 
Als gevolg is het niet meer mogelijk om het onderscheid te maken tussen takken van een conditionele sprong op basis van de gemeten instructietimings.
\textit{Binary rewriting} is het aanpassen van een gecompileerd programma zonder gebruikt te maken van de originele code. 
Omdat deze oplossing niet berust op het aanpassen van de hardware, of op het hercompileren van de code, is het mogelijk om deze toe te passen op bestaande 'off-the-shelf' programma's. 

Het voorgestelde algoritme manipuleert het programma door aanpassingen uit te voeren op de \textit{Control Flow Graph} (CFG). 
Dit is een datastructuur die de \textit{control flow} van een programma voorstelt, bestaande uit knopen en zijden. Hierbij bevat elke knoop een verzameling van instructies die telkens 
als een geheel uitgevoerd worden. De zijden geven aan hoe de \textit{control flow} van het programma spring tussen de verschillende noden. 
De CFG moet voldoen aan twee eigenschappen vooraleer de nodige instructies toegevoegd kunnen worden. 
Ten eerste moeten alle paden naar een gegeven node dezelfde lengte hebben, en ten tweede 
moeten de lengtes van alle paden naar een blad in de CFG even lang zijn. Een blad is een node met geen uitgaande zijden. 
De eerste fase van het algoritme vult de CFG met extra knopen aan zodat de graaf voldoet aan bepaalde eigenschappen. 
De tweede fase voegt dan niveau-gewijs instructies toe aan de knopen van de CFG om zo de \textit{timing leaks} te sluiten. 

Het algoritme is ge\"implementeerd voor de Intel x86\_64 architectuur aan de hand van RetroWrite, een \textit{binary rewriting} framework. 
Een aantal experimenten zijn uitgevoerd op een verzamling van testprogramma's om te bepalen of het algorithme doeltreffend is in het sluiten van de \textit{timing leaks}, of het dit doet 
zonder de output van het programma te veranderen, en om de impact op de performantie van het programma te evaluaren. 
Uit de resultaten blijkt dat het algoritme correct alle \textit{timing leaks} kan dichten. Het doet dit ook zonder de uitkomst te veranderen bij alle programma's, met \'e\'en uitzondering. 
Deze uitzondering bevat een functie die opgeroepen wordt in \'e\'en tak van een conditionele sprong, waarbij de functie ook nog de globale staat van het programma aanpast. Het algoritme kopieert deze oproep naar de andere tak, waardoor
het de globale staat van het programma gewijzigd wordt waar dit voorheen niet gebeurde. 

Het algoritme voorgesteld in deze thesis is doeltreffend in het sluiten van \textit{timing leaks}. 
In tegenstelling tot eerdere oplossing voor het sluiten van \textit{timing leaks} kan het toegepast worden op bestaande, 'off-the-shelf' binaire bestanden, omdat het gebruikt maakt van \textit{binary rewriting}.  
Verder kan het algoritme ge\"implementeerd worden voor een architectuur indien er hiervoor een \textit{binary rewriting} framework bestaat. 
\end{abstract*}


\listoffiguresandtables


\mainmatter 


\lstdefinelanguage
   [x64]{Assembler}     % add a "x64" dialect of Assembler
   [x86masm]{Assembler} % based on the "x86masm" dialect
   % with these extra keywords:
   {morekeywords={CDQE,CQO,CMPSQ,CMPXCHG16B,JRCXZ,LODSQ,MOVSXD, %
                  POPFQ,PUSHFQ,SCASQ,STOSQ,IRETQ,RDTSCP,SWAPGS, %
                  rax,rdx,rcx,rbx,rsi,rdi,rsp,rbp, %
                  r8,r8d,r8w,r8b,r9,r9d,r9w,r9b, %
                  r10,r10d,r10w,r10b,r11,r11d,r11w,r11b, %
                  r12,r12d,r12w,r12b,r13,r13d,r13w,r13b, %
                  r14,r14d,r14w,r14b,r15,r15d,r15w,r15b}} % etc.


\chapter{Introduction}
\label{cha:introduction}
As a result of the increasing popularity of IoT devices and cloud-computing services, software is often executed on third party platforms. 
Embedded devices often support software extensibility, allowing users to install additional software onto their device. 
Large cloud-service providers such as Amazon and Microsoft allow users to run virtually any piece of software on their devices. 
To ensure that the various components on these systems are isolated from each other, a sizeable software layer is often introduced in the form of an operating system or 
a hyper visor.
Unfortunately, however, this software layer is very difficult to get sufficiently secure \cite{psma}.
As a result it has become increasingly important to protect software from attacks even in the presence of a compromised system. 
One line of research is the use of Protected Module Architectures (PMAs). 
These are architectures that enforce isolation of components directly in hardware.
PMAs ensure that the untrusted OS cannot access the data or code of a protected module \cite{Nemesis}. 

Recent research has shown, however, that it is still possible to extract secret data from protected modules through the use of controlled-channel attacks. 
Controlled-channel attacks are a type of side-channel attack that leverage an attacker's control of the system to extract more information \cite{Xu}.	
Side-channel attacks aim at extracting secrets from a system through measurement of physical parameters \cite{side-channel}. 
In the case of controlled-attacks an attacker has increased capabilities for extracting data, since he can use system events such as as page faults, scheduling decisions, and interrupts to open additional side-channels. 
Research has shown that these attacks are able to extract information even from modules that are protected by PMAs. 

Recently Van Bulck et al. \cite{Nemesis} have been able to exploit the CPU interrupt mechanism to leak micro-architectural instruction timings from protected modules. 
Their attack, called Nemesis, exploits the property that all arriving interrupts are only served after the current instruction is done executing. 
As a result, the interrupt latency, the delay between arrival of the interrupt and the execution of the first instruction in the interrupt service routine, increases with the number of cycles left to execute.
An attacker with control over the system can exploit this by carefully timing interrupts and measuring the interrupt latency to infer the duration of the interrupted instruction. 

Van Bulck et al. \cite{Nemesis} convincingly demonstrate that it is possible to use these Nemesis-type interrupt attacks to leak information about secret-dependent control flow of the program. 
The authors require two branches of a conditional jump that contain instruction with different execution times. 
If the control flow depends on a secret, the attacker is able to infer some information about it. 

A number of countermeasures have been proposed for closing timing side-channels, both software-based approaches and hardware-based approaches. 
Hardware based-approaches aim to close side-channels by modifying the architecture. 
A limitation of hardware-based approaches is that they require the replacement or modification of existing devices. This makes them difficult
to apply in the field. 
Software-based approaches are implemented at the language level. A number of transformations have been proposed that can automatically close timing leaks \cite{programcounter}. 
Additionally tools exists that can verify if a program is safe from leaks \cite{verify-constant-time, Barthe}. Unfortunately these solutions often require recompilation of the program. This means
that the source code of the program has to be available. As a result these solutions are not generally applicable to commercial off-the-shelf binaries. 

The solution proposed in this thesis makes use of binary rewriting to circumvent this issue. Binary rewriting is the alteration and transformation of a compiled program without having the source code at hand \cite{rewriting-survey}. 
The target binary is decompiled and reconstructed into an assembly file and additional dummy instructions are inserted into the program to close any timing leaks. 
After applying all transformations the assembly file can then be compiled into an executable using any existing compiler. 
Unlike previous solutions this algorithm can be applied to commercial off-the-shelf binaries. 

\section{Thesis Goal and Outline}
This paper presents a novel algorithm for automatically transforming a program in order to remove any timing leaks. It achieves this by addressing the core cause of the vulnerability: differences in 
latencies between corresponding instructions in secret-dependent branches. Corresponding instructions are instructions that are the same distance away from a branching instruction. 
The proposed algorithm inserts additional instructions such that all corresponding instructions have the same latency without changing affecting the program output. 
Unlike previous solutions that require recompilation, the proposed algorithm transforms the target program through binary rewriting. 

The main contributions of this paper are:
\begin{enumerate}
\item The paper presents a novel algorithm for automatically transforming a program to remove any timing leaks. Unlike previous solutions it is applicable to off-the-shelf binaries. 
\item The paper presents an implementation of this algorithm for the Intel x86-64 architecture. 
\item The paper presents an evaluation of the algorithm based on a suite of benchmark programs. 
\end{enumerate}

Chapter \ref{cha:background} will provide additional background information on PMA's, the Nemesis attack, and binary rewriting. 
Chapter \ref{cha:design} outlines the design of the proposed algorithm, and formalizes the property it aims to enforce.  
Aspects specific to the implementation of the algorithm are further discussed in chapter \ref{cha:implementation}. 
A number of experiments have been designed and run to evaluate if the proposed algorithm is effective and correct. 
Chapter \ref{cha:evaluation} describes the setup of these experiments and discusses the results. 
Next a number of related works is discussed in chapter \ref{cha:evaluation}.
Finally chapter \ref{cha:conclusion} provides some discussions about the benefits and limitations of the proposed algorithm and suggests 
future work. 
\chapter{Background}
\label{cha:background}
This chapter aims to provide additional information on some concepts related to the thesis. 
Section \ref{sec:pma} describes protected module architectures and the motivation behind them. 
Section \ref{sec:nemesis} outlines the workings of the Nemesis attack, providing an example to further illustrate it. 
Finally section \ref{sec:rewriting} introduces the concept of binary rewriting and discusses its uses. 
\section{Protected Module Architectures}
\label{sec:pma}
Because of the increasing popularity of IoT devices, more and more embedded computing devices are being connected to the Internet. 
These devices are often more susceptible to being exploited because they support software extensibility. 
Additionally, because these devices are connected to a network, the risk increases since attacks can be done remotely. 

An important technique for securing such devices is hardware-supported virtual memory and processor privilege levels.
The OS can build on this support to isolate a process from any other malicious processes on the device. 
However, this introduces a sizable software layer however which is difficult to get sufficiently secure \cite{psma}.
If the attacker controls the OS then its capabilities for attacking a process on the devices increase considerably. 

Maene et al. \cite{trusted-computing-architectures} state that \textit{the goal of trusted computing is to develop technologies which give users guarantees about the behaviour of the software running on their devices}.
An important aspect of trusted computing is therefore to protect software even when attackers have full control of the system. 
One means of achieving this is through the use of Protected Module Architectures (PMAs). 
These architectures seperate critical components into protected modules, also called enclaves,
that are isolated from one another through hardware.  

A number of Protected Module Architectures (PMA) have been developed to address this problem, both by researchers and industry. 
PMAs have been developed for both low-end microcontrollers found in embedded systems \cite{trustlite, smart} as well as high-end processors \cite{isox}.
One architecture developed for embedded systems is Sancus. 
Sancus is a security architecture that can provide strong isolation guarantees on networked embedded systems, 
and has been implemented on a modified TI MSP430 micro-controller \cite{sancus}. 
At the higher end of the spectrum, there is Intel SGX.
Intel SGX is an extension  added to the Intel architecture that allows applications to instantiate enclaves. 
Enclaves are areas in the application's memory that are protected from access from outside of the enclave, even from 
privileged software such as the OS \cite{SGX}. 

Research has shown that it is still possible to extract information from protected applications in PMAs. In their work Xu et al. \cite{Xu} introduce a novel type of side-channel attack 
called controlled-attacks. These attacks are categorized by untrusted operating systems that create side-channels through its extensive control of the system.
The authors were able to leverage the OS' high degree of control over the system to attack applications that were previously out of reach of side-channel attacks, and were able to 
extract large amounts of data in a single run. 

\section{Nemesis Side-Channel Attack}
\label{sec:nemesis}
More recently, Van Bulck et al. \cite{Nemesis} developed Nemesis, a controlled-channel attack that leverages the interrupt mechanism to extract sensitive information from 
enclaved applications. The authors were able to exploit timing differences in the latency between the arrival of an interrupt request (IRQ) and the execution of the first instruction in the 
interrupt service routine (ISR). They state that their attack is \textit{based on the key observation that an IRQ during a multi-cycle instruction increases the interrupt 
latency with the number of cycles left to execute}. By carefully and deliberately interrupting a process at the right time the authors were able to infer the duration of the interrupted instruction. 
Potential attackers can use this information to determine where the instruction is situated in the program's control flow. When the instruction is part of a secret-dependent branch, the 
attacker is able to infer some information about the secret, successfully leaking sensitive information from the program. Van Bulck et al. \cite{Nemesis} showed that this attack is applicable to 
the whole computing spectrum. They were able to apply their attack to the aforementioned Sancus architecture as well as Intel SGX enclaves.  

Figure \ref{fig:pseudo-assembly} illustrates how such an attack might work with a piece of assembly pseudocode. An attacker who is in control of the OS could carefully interrupt the program right
after the conditional jump at line 5. Depending on the value of register r1 the next interrupted instruction is either the addition instruction at line 4 or the multiplication instruction at line 7. 
By measuring the interrupt latency the attacker can infer which of the two instructions was being executed at the time of the interrupt and, more importantly, infer if the value in register r0 is equal to 0. 



\lstset{language=[x64]Assembler, numbers=left, stepnumber=1, frame=single}
\begin{figure}

    \begin{lstlisting}
	CMP r1, $0
	JEQ .l1
	.l1: 
	ADD r1, r2 		; 1 cycle instruction
	JMP .end
	.l2: 
	MUL r1, r2 		; 2 cycle instruction
	JMP .end
	\end{lstlisting}
	\caption{Assembly pseudo-code with a secret-dependent branch that is vulnerable to Nemesis attack}
	\label{fig:pseudo-assembly}
\end{figure}


\section{Binary Rewriting}
\label{sec:rewriting}
Binary rewriting is the alteration of a compiled program without having the source code at hand. 
Applications of binary rewriting include observing programs during execution, optimizing programs using run-time patching, and 
hardening applications against attacks. In the case of dynamic binary rewriting the rewriting happens during execution of the program. 
Static binary rewriting, on the other hand, occurs before the binary is executed \cite{rewriting-survey}. 
Binary rewriting tools have been developed for both low-end architectures found in embedded devices \cite{microsbs} as well as high-end architectures found in home computers and servers 
\cite{ instruction-punning, Dinesh2020RetroWriteSI, E9Patch}. 

\chapter{Design}
\label{cha:design}

\section{Introduction}

The root cause of the vulnerability exposed by Nemesis-style attacks are differences in the latencies of two instruction that occur at the same 
position in two branches of a secret-dependent branching instruction. 
In practice an attacker can exploit this vulnerability by generating latency traces along different paths of the program's control flow. 
Any differences in instruction latencies will be reflected as differences in these latency traces. 
By carefully inspecting the relevant sections of the latency traces the attacker can infer which paths were taken for a given input. 
In cases where the path depends on some secret data the attacker is then able to infer information about this data, successfully leaking information from the program \cite{Nemesis}.

The goal of the algorithm outlined in this section is to ensure that latency traces cannot be used to leak information in this way.
It does this by inserting additional instructions into branches of a secret-dependent branching instruction. These instructions 
are carefully selected such that they have the same latency as their corresponding instruction in the other branch. This ensures that
any instructions that occur at the same position in two different branches have the same latency. As a result the sections of latency traces
that correspond to these branches will be identical, making it impossible for an attacker to infer information. 


The proposed algorithm inserts additional instructions into the functions of a program through manipulation of the function's control flow graph (CFG). 
This graph consists of nodes and vertices, where each node contains a sequence of instructions. 
One of the main operations performed on the graph is the alignment of a set of nodes. 
This operation consists of inserts additional instructions into nodes such that all instructions at a given position in any of the nodes have the same latency. 

Not all structures found in a control flow graph are suitable for alignment. The aforementioned alignment operation therefore has some conditions on the structure of the control flow graph that need to be met. 
The other main operation of the algorithm consists of inserting additional nodes into the graph such that these conditions are met. 

Section \ref{sec:property} will formally define the property that needs to hold for a program in order for Nemesis-style attacks to be mitigated. Section \ref{sec:cfg} 
introduces the CFG data structure and translates the aforementioned property to such structures. Finally, sections \ref{seq:equalising} and \ref{seq:alignment} describe the insertion and alignment of nodes, respectively. 

\section{Nemesis-sensitive property}\label{sec:property}
In their paper Pouyanrad et. al have formally defined the Nemesis-Sensitive property.  Let $region^{then}(ep)$ and $region^{else}(ep)$ capture the set of execution points belonging to the branch target and the other region of some branching instruction $ep$. Let $ep^i$ be the i'th instruction in a region. A program P with a secret-dependence branch in $ep$ and $region^{then}(ep)$ 
and region $region^{else}(ep)$ with the same number of execution points, satisfies the nemesis-sensitive property if and only if:  

\begin{equation} \label{eq:nemesisProperty}
    \begin{split}
    \forall ep^i \in region^{then}(ep) : \forall ep^j \in region^{else}(ep) \text{ \textit{such that} } i=j :  \\ 
    (s_{ep^i} \xrightarrow[]{t} s_{ep^i_{next}}) \land (s_{ep^j} \xrightarrow[]{t'} s_{ep^j_{next}}) \iff t = t' \\ 
    \end{split}
\end{equation}
\cite{MSP430Detection}

The relation $s \xrightarrow[]{t} s'$ models the transition between program states $s$ and $s'$, declaring that the transition between $s$ and $s'$ takes a time $t$. 
For a given instruction this time $t$ is equal the instruction's latency. This property states that for any two corresponding instructions in the branches their latencies should be the same.

If this nemesis-sensitive property holds for a program then an attacker is not able to infer which branch was taken by the program based on latency traces of the program.

\section{CFG}\label{sec:cfg}

The Control Flow Graph (CFG) is a data structure that represents the control flow of a function. 
If a program has multiple functions then each function is represented by a separate CFG. 
A CFG consists of nodes $V$ and directed edges $E$. Each node $V$ contains a 
contiguous sequence of instructions that is always executed as a whole. This implies that a branching instruction can only be found at the end of a node, 
and an instruction that is the target of a branching instruction can only occur at the start.

An edge is drawn from node $v$ to node $v'$ if and only if the last instruction in $v$ can be followed by the first instruction in $v'$ 
when following program control flow. The algorithm only considers branching instructions that are binary in nature, so a node in the CFG can have at most 2 successors. 
A node is said to be secret-dependent if its last instructions is a 
secret-dependent branching instruction. 

Each node has a latency sequence associated with it. This latency sequence is equal to the latencies of the node's instructions.  
A latency trace along a path of the CFG is then equal to the concatenation of the latency sequences of each node along the path.
Figure \ref{fig:exampleCFG} shows an example of a such a CFG, along with the original program it is created from. The CFG also contains the latency for each instruction. Note that by convention the only node with no incoming edges is considered the starting node of the CFG. 

\begin{figure}
\centering
\begin{subfigure}{.4\textwidth}
  \centering
  
    \begin{lstlisting}[language=C]
int main(){
        int a = 10; 
        int b = 20; 
        if (a < b){
            int temp = b; 
            b = a; 
            a = temp; 
        } 
        return a;  
}\end{lstlisting}
  \caption{C program}
  \label{fig:c_program}
\end{subfigure}%
\begin{subfigure}{.7\textwidth}
  \centering
  \includegraphics[width=.7\textwidth]{images/sample_program_graph.png}
  \caption{Corresponding CFG}
  \label{fig:c_program_cfg}
\end{subfigure}
\caption{Example program with corresponding CFG}
\label{fig:exampleCFG}
\end{figure}

Following the property described in section \ref{eq:nemesisProperty}, the nemesis-sensitive property can be defined for a node in the CFG. Let $v$ be a secret-dependent node. 

Let $v_f$ be a node such that all paths from $v$ to some leaf go through $v_f$. Then $region^{then}(v)$ can be defined as the set of nodes reachable following the 
first of $v$'s outgoing edges up to and including $v_f$ and $region^{else}(v)$ as the set of nodes reachable following the other outgoing edge up to and including $v_f$.

Any differences in the latency sequences of two nodes can only be used to infer which branch was taken at the nearest branching point that is an ancestor of both nodes. 
Any differences in latencies between two nodes that are descendants of $v_f$ can therefore only be used to infer information about which branch was taken at $v_f$. 
This means that all nodes below $v_f$ do not have to be considered. If no such node $v_f$ exists then the regions simply consist of all nodes reachable from $v$ through one of its outgoing edges. 

The depth of $region(v)$ is defined as being the length of the longest path from $v$ to some node $v' \in region(v)$ that does not contain a cycle.
Let $n^i \in region(v)$ be a node such that there is a path going to it from node $v$ of length $i$. 
A secret-dependent node $v$ and $region^{then}(v)$ and $region^{else}(v)$ with the same depth satisfies the nemesis-sensitive property if and only if 
\begin{equation} \label{eq:nemesisPropertyNode}
    \begin{split}
    \forall n^i \in region^{then}(v) : \forall n^j \in region^{else}(v) \text{ \textit{such that} } i=j :  \\ 
    latencies(n^i) = latencies(n^j)
    \end{split}
\end{equation}
where $latencies(n)$ is a function mapping a node $n$ to its latency sequence. This property states that the latency sequence of any two nodes that are the same distance away 
from some secret-dependent node need to have identical latency sequences. If this property holds then the critical sections of latency traces will be identical and cannot be used to 
infer information about the secret-dependent branch. 

Figure \ref{fig:regionExamples} illustrates how the borders of each region are defined. The secret-dependent node is marked in green, while the two branches are marked in red and blue. In the second example, the node marked in purple belongs to both regions. In example \ref{fig:regionExampleA} there is no node such that all paths from the secret-dependent node to a leaf go through it, so the regions extend all the way to the leaves. In example \ref{fig:regionExampleB} all paths that start in the secret-dependent node go through the node 6. Any differences in nodes 7 and 8 can only be used to infer information about the branch in node 6. These nodes therefore do not have to be considered.  


\begin{figure}
 \centering
 \subfloat[]{\includegraphics[height = 6cm]{images/nemesis-property-example-1.png}\label{fig:regionExampleA}}
 \subfloat[]{\includegraphics[height = 6cm]{images/nemesis-property-example-2.png}\label{fig:regionExampleB}} 

 \captionof{figure}{then-else regions for secret-dependent nodes}
 \label{fig:regionExamples}
\end{figure}


\section{Equalising}\label{seq:equalising}
There are 2 structures found in a program's control flow graph that make it impossible to enforce the nemesis-sensitive property for a node as defined in the previous section. 
These structures are shown in figure \ref{fig:problemstructures}.  
The first stage of the algorithm consists of inserting nodes such that these structures no longer occur. 

The first such structure occurs when a function contains some sequence of instructions that is only executed if some condition is true and is illustrated in figure \ref{fig:optional}.
Analogously the corresponding CFG will contain a conditional node that is only reached if the condition is true.
A consequence of this is that there will be some node in the CFG that has at least two paths to it. One path will contain the conditional node while the other will not. 
Additionally, the path with the conditional node will be longer than the other path. 


In such cases it is impossible to align the latency traces of the two paths since of the latency traces will always be longer than the other one. 
Because the nodes in the shorter path form a subset of the nodes in the longer path it is impossible to modify the shorter path without also modifying the longer path, making it 
impossible to ever equalize the lengths of the traces. 

The second problematic structure occurs when one of the branches is shorter than the other one, as shown in figure \ref	{fig:unequal}. In such cases there will be some nodes in 
one branch that have no corresponding nodes in the other branch, making it impossible to align them. 

The nemesis-sensitive property as defined in section \ref{eq:nemesisPropertyNode} entails that it is impossible for a node to satisfy the property if one of these 
structures occurs in its branches, since in both cases the regions have different depths. 
The first stage of the algorithm therefore consists of first equalizing all path lengths and then 
equalizing all branch depths. This is achieved by inserting additional instructions into the CFG.  
Algorithms \ref{alg:equalizePaths} and \ref{alg:equalizeBranches} depict pseudo-code for equalizing paths lengths and equalizing branches respectively. 

\subsection{Extract Sub-graph}
The different procedures described in this section only need to take into account the branches of secret dependent nodes. 
These branches correspond to the regions $region_{then}(v)$ and $region_{else}(v)$ as defined in section \ref{sec:cfg}.
The procedure \textit{ExtractSubgraph}, shown in algorithm \ref{alg:equalizePaths}}, extracts the subset of the graph that contains only the nodes that belong to 
either one of these regions for a given secret-dependent node. 
The edges of this new CFG are all the edges of the original CFG whose head and tail are a part of this subset. 

To determine which nodes are a part of this subgraph all immediate dominators are determined starting from node $n$. 
A node $u$ is said to dominate another node $w$ with respect to $n$ if every path from $n$ to $w$ passes through 
$u$. 
Node $v$ is the immediate dominator of $w$ if $v$ dominates $w$ and every other dominator of $w$ dominates $v$ \cite{dominator}.
The immediate dominator is determined for each node reachable from $n$. If all leaves reachable from $n$ have the same immediate dominator $d$ then all paths from $n$ to some leaf go through $d$. 
In this case any descendants of $d$ are not part of $region_{then}(v)$ or $region_{else}(v)$ and do not have to be included in the sub-graph. 

If such a node $d$ exists then the nodes that are a part of the sub-graph are all nodes that are on a path from $n$ to $d$. If $d$ does not exists 
the sub-graph nodes are all nodes that are on a path from $n$ to some leaf. This definition is analogous to the definition for $region_{else}(v)$ and $region_{then}(v)$ as defined in section \ref{sec:cfg}.

\subsection{Equalize path lengths}

To equalize all path lengths starting from some secret-dependent node $v$, first a subset of the graph's nodes are extracted such that 
only the regions $region^{then}(v)$ and $region^{else}(v)$ are considered.  
Next the length of the longest path is computed from $v$ to all nodes in the sub-graph. 

Let $(u, v)$ be an edge in the sub-graph. Let $d(u)$ and $d(v)$ be the length of the longest path to $u$ and $v$. If the difference between $d(u)$ and $d(v)$ is more than 
one then there exist at least two paths to $v$. The first path goes through $u$ and has length $d(u)+1$. The second path goes through a different predecessor of $v$ and has 
length $d(v)$. 

The solution is based on the observation that the edge $(u, v)$ is a part of the shorter path. Additional nodes can be inserted between $u$ and $v$ to increase the length of the path until it 
is as long as the longest path. 
All other paths to $v$ will be unaffected. 

The procedure for equalizing the path lengths iterates over all edges of the sub-graph. If the distances to $u$ and $v$ differ by more than one then nodes are inserted 
into the edge between $u$ and $v$ such that the path that goes to $v$ through $u$ is of the same length as the longest path to $v$. 

\subsection{Equalize branches}
The branches of the CFG can be equalized in a similar way. Given some secret-dependent node $v$ a subset of the graph's nodes are extracted. 
The lengths of the longest paths are computed for all nodes in the sub-graph. 
The maximum path length is then determined as being the longest path length to one of the leaves of the CFG.
The procedure then iterates over all leaves in the sub-graph and determines the difference between the distance to the leaf and the maximum path length. 
If this difference is larger than zero then additional nodes are inserted as the predecessor to the leaf until the distance to the leaf is equal to the maximum 
path length. Because the final instruction in a leaf is a return statement any new nodes have to be added as predecessors. 
 
\begin{figure*}[t!]
    \centering
    \begin{subfigure}[t]{0.5\textwidth}
        \centering
        \includegraphics[height=6cm]{images/optional.png}
        \caption{optional node}
        \label{fig:optional}
    \end{subfigure}%
    ~
    \begin{subfigure}[t]{0.5\textwidth}
        \centering
        \includegraphics[height=6cm]{images/unequal.png}
        \caption{unequal branches}
        \label{fig:unequal}
    \end{subfigure}
    \caption{problematic structures in CFG}
    \label{fig:problemstructures}
\end{figure*}

\section{Alignment} \label{seq:alignment}

During the alignment stage the nodes of the CFG are aligned in a level-wise manner. 
The alignment of a set of nodes consists of inserting instructions such that all instructions at a given position across all nodes in the set have the same latency. 
The level of a node is defined as being the distance between the root of the graph and the node. The first stage of the algorithm ensures that all paths to a given nodes have the same length making the level of a node a well 
defined value. The alignment stage iterates over all the levels of the sub-graph and aligns the set of nodes found at that level. Algorithm \ref{alg:align} depicts pseudocode for this stage of the algorithm.

\subsection{Basic Operation}
The core of the alignment operation consists of repeatedly selecting a reference node and inserting instructions into the other nodes to match the latencies of the reference. 
Each iteration a set of candidate nodes is determined, from which the reference node is then selected. 

An index variable $i_{ref}$ is used to keep track of the position of the first instruction that has not yet been aligned. This variable is initially equal to zero and is incremented every iteration. 
The instruction at position $i_{ref}$ in the reference node is called the reference instruction.

The algorithm iterates over all nodes that are not the reference node and verifies if the instruction at position $i_{ref}$ has the same latency. 
If the two latencies are not equal or if the node is shorter than the reference node a new instruction is inserted at position$i_{ref}$. 
The latency of this new instruction is equal to the latency of the reference instruction. 
Once this has been repeated for all nodes in the set then all instruction with at position $i_{ref}$ have the same latency and the variable can be incremented. 

\subsection{Selecting the Reference Node}
Because an instruction is potentially added to each node that is not the reference node the reference node needs to have at least as many instructions as the node with the largest number of instructions. 
This ensure that at some point all nodes have the same number of instructions. 
The set of candidate nodes therefore consists of all nodes that have $n_{max}$ instructions, where $n_{max}$ is the number of instructions in the longest node. 

The instruction in the reference node at position $i_{ref}$ determines what instructions are inserted into the other nodes. 
There are some restrictions on the selection of the reference node. 
These stem from the fact that branching instructions are special cases that will result in the insertion of branching instruction in the other nodes. 
Additionally a branching instruction cannot be inserted into the middle of a node since this will change the program's control flow. 
A node can therefore not be selected as the reference node when the instruction at position $i_{ref}$ is a branching instruction, unless during the very last iteration. 

If among the set of candidate nodes there is at least one node with a non-branching instruction at position $i_{ref}$ then this node is selected as the reference node. 
If there are multiple such nodes then a candidate node can be selected arbitrarily. 
The case where all candidate nodes have a branching instruction at position $i_{ref}$ can only occur during the very last iteration. 
At this point any of the candidate nodes are suitable and one is selected arbitrarily again. 

\subsection{Constructing NOP instruction}
\label{sec:nop}
Any instructions that is inserted into the program can have no effect on the program outcome. In this regard they are the same as the no-operation instruction and are 
referred to as NOP instructions. For each latency class a template NOP instruction has been determined. 
For some latency classes a NOP instruction exists that has no effect on the program state. These can be inserted into the program as-is. 
For other latency classes the NOP instruction modifies some register value. In these cases the algorithm selects a registers that can safely be used. 
This needs to be a register that is not in use at the time of execution of the instruction in order to guarantee that the program outcome is not changed. 

There are two types of free registers. 
A register can be free because its current value is no longer used. This occurs when the register is overwritten at some later point without being read first. 
Alternatively a register can be free because it isn't used anywhere in the current function. 
In the latter case, however, it is possible that the register is in use by the caller, since there is no guarantee that the caller stored all the registers it uses. 

The function is statically analyzed to determine which registers are free to use for this purpose. 
If a register of the first type exists then it can be used as the operand of the NOP instruction and the resulting instruction 
can be inserted as-is into the node. If no such registers exists, a free register of the second type is selected. 
In this case additional instructions are inserted into the program to ensure that the original value of the register is not lost. 
In the root of the CFG additional instructions are inserted to push the register value onto the stack, while in every leaf instructions are inserted 
that pop the value from the stack. Once these instructions have been inserted the register effectively becomes a free register of the first type and can 
later be reused when construction additional NOP instructions.

If there are no free registers available, any register is arbitrarily selected. Additional instructions are inserted before and after the NOP instruction to push and pop the register value. To ensure the nodes are 
still balanced these push and pop instructions are inserted across all nodes of the current level. 

If the reference instruction is a branching instruction the the NOP instruction will also be a branching instruction. The target of the branching instruction is then the address of the first instruction of the node's successors. 

If the reference instruction is a call to a function then the NOP instruction will be  a call to the same function. The instruction is simply duplicated into the current node. 
If the function contains no secret dependent node then this ensures that any latency traces cannot leak information from the program. Otherwise the function that is called needs to be aligned as well. 
It is only safe to insert a call to a function this way if the function in question has no effects on the program state. If the function does modify the program state then inserting additional calls can 
result in erroneous program outputs. 

\section{The algorithm}


\subsection{Cycles}
The operations described in section \ref{seq:equalising} require the CFG to be acyclic.
To account for this all cycles are removed from the CFG beforehand and later restored. 
The edge that needs to be removed is determined based on the depths of the nodes in the cycle. 
The depth of a node is defined as being the length of the longest path to the node from the root. 
Removing the cycle is done by removing the edge from that graph that connects the deepest node to the most shallow node. 
The tail and head are stored for all edges that are removed so that they can later be restored. 

The removal of these edges has no effect on the operations of the algorithm if the cycle is not part of a branch of a secret dependent node.  
Cycles within a branch of a secret dependent node are not supported by the proposed algorithm.
In this case the algorithm correctly aligns the nodes of the branches but the latency traces will still not be identical.
The branch that contains the cycle will be executed a higher number of times, resulting in a longer latency trace. 

\subsection{Closing timing leaks}
Algorithm \ref{alg:closeleaks} illustrates the top-level operation of the algorithm. 
The secret-dependent branching instructions are passed to the algorithm as arguments. 
Determining which branching instructions are secret-dependent is not part of the proposed algorithm. 
The user has to provide the algorithm with the address of the target instruction.

Based on the given instructions the secret-dependent nodes of the CFG are determined. 
These are all nodes that contain a secret-dependent instruction. 
First the equalizing operations are applied for each secret-dependent node. 
Only once all necessary nodes have been inserted is the CFG aligned.
This order of operations ensures that the algorithm works correctly even when secret-dependent node is nested inside the branch of another secret-dependent node. 
This can occur when, for example, a program contains a nested if-statement. Cycles are removed from the CFG and restored as described in the previous section.


\begin{algorithm*}
  \SetAlgoLined
  \DontPrintSemicolon
  
  \SetKwFunction{equalizePathsFnName}{EqualizePathLengths}
  \SetKwFunction{extractSubgraphFNName}{ExtractSubGraph}
  \SetKwFunction{computeLongestPathFNName}{ComputeLongestPathLengths}
  \SetKwFunction{CreateNodeFNName}{CreateNode}
  \SetKwFunction{InsertNodeFNName}{InsertNodeBetween}
  \SetKwFunction{edges}{Edges}
  \SetKwFunction{nodes}{Nodes}
  \SetKwFunction{leaves}{Leaves}
  \SetKwFunction{topologicalorder}{TopologicalOrder}
  \SetKwFunction{successors}{Successors}
  \SetKwFunction{predecessors}{Predecessors}
  \SetKwFunction{addNode}{AddNode}
  \SetKwFunction{addEdge}{AddEdge}
  \SetKwFunction{removeEdge}{RemoveEdge}
  \SetKwFunction{max}{Max}
  \SetKwProg{equalizePaths}{Procedure}{}{}
  
   \equalizePaths{\equalizePathsFnName{g: CFG, v: Node}}{
   \textit{subgraph} $\leftarrow$ \extractSubgraphFNName(g, v) \\  
   \textit{longestPathLengths} $\leftarrow$ \computeLongestPathFNName{subgraph, v} \\
   \ForAll{(u, v) $\in$ \edges{subgraph}}{
   diff $\leftarrow$  \textit{longestPathLengths}[u] - longestPathLengths[v] \\
   \uIf{diff $>$ 1}{
        \textit{head} $\leftarrow$ v \\ 
        \For{$i \in$ 1, 2, ..., diff-1}{
%            \FSetLatencies(node, target\_latencies)
            \textit{newNode} $\leftarrow$ \CreateNodeFNName{} \\
            \InsertNodeFNName{newNode, u, head} \\
            \textit{head} $\leftarrow$ newNode \\
        }
   }
   }
   }{}
   
     \SetKwProg{ComputeLongestPathLengths}{Function}{}{}
  \ComputeLongestPathLengths{\computeLongestPathFNName{g: CFG, s: Node}}{
  dist = \{n : -1 $\vert$  n $\in$ \nodes{g} \} \\
  dist[s] $\leftarrow$  0 \\
  \ForAll{n $\in$ \topologicalorder{g}}{
    
    \ForAll{succ $\in$ \successors{n}}{
        dist[succ] $\leftarrow$ \max{dist[succ], dist[n] + 1} \\
    }
    
  }
  \KwRet dist \\
  }
  
   \SetKwFunction{computeImmediateDominators}{computeImmediateDominators}
   \SetKwFunction{computeAllPaths}{computeAllPaths}
  \SetKwProg{ExtractSubGraph}{Function}{}{}
  \ExtractSubGraph{\extractSubgraphFNName{g: CFG, n: Node}}{
  \textit{immedidateDominators} $\leftarrow$  \computeImmediateDominators{g, n} \\
  \textit{leafDominators} $\leftarrow$  \{ $u$ $\in$ \nodes{g} $\vert(u, v) \in$ \textit{leafDominators} $ \land v \in$ \leaves{g} \} \\
  
  \uIf{$\vert$ leafDominators $\vert$ = 1}{
    dominator $\leftarrow$  leafDominators[0] \\
  } \uElse{
    dominator $\leftarrow$  $\varnothing$ \\
  }
  
  \textit{subgraphNodes} $\leftarrow$  \{ \} \\
  \uIf{domintor $\neq \varnothing$}{
  	\ForAll{$path \in$ \computeAllPaths{g, n, dominator}}{
  		$subgraphNodes \leftarrow subgraphNodes \cup (\{ p \vert p \in path \} \setminus subraphNodes)$
  	}
  } \uElse{
    \ForAll{l $\in$ \leaves{g}}{
        \ForAll{$path \in$ \computeAllPaths{g, l, dominator}}{
  		$subgraphNodes \leftarrow subgraphNodes \cup (\{ p \vert p \in path \} \setminus subraphNodes)$
  	}
    }
  }
  
  $subgraphEdges \leftarrow \{ (u, v) \vert u \in subgraphNodes \land v \in subgraphNodes \land (u, v) \in $ \edges{g} $\}$

	\KwRet $(subgraphNodes, subgraphEdges)$
  }
  
  \caption{Equalize Path Lengths}
  \label{alg:equalizePaths}
\end{algorithm*}

\begin{algorithm*}
    \SetAlgoLined
    \DontPrintSemicolon
    \SetKwProg{equalizeBranches}{Procedure}{}{}
    \SetKwFunction{equalizeBranchesFNName}{EqualizeBranches}

    \equalizeBranches{\equalizeBranchesFNName{g: CFG, n: Node}}{
    \textit{subgraph} $\leftarrow$ \extractSubgraphFNName(g, v) \\  
    \textit{longestPathLengths} $\leftarrow$ \computeLongestPathFNName{subgraph, v} \\
    \textit{maxPathLength} $\leftarrow$ \max{$\{ longestPathLengths[v] \, \vert \,  v \in  \leaves{subgraph}\}$} \\
    \ForAll{leaf $\in$ \leaves{subgraph}}{
        diff $\leftarrow$ \textit{longestPathLengths[leaf]} - maxPathLength \\
        
        \uIf{diff $>$ 0}{
                $v$ $\leftarrow$ leaf \\
                \For{i $\in$ 1, 2, ..., diff}{
                    newNode $\leftarrow$ \CreateNodeFNName{} \\
                    \addNode{g, newNode} \\
                    \ForAll{$p \in$ \predecessors{v}}{
                        \addEdge{g, (p, newNode)} \\
                        \removeEdge{g, (p, v)} \\
                    }
                    \addEdge{g, (newNode, v)} \\
                }
        }
    }
    }
    \caption{Equalize Branches}
    \label{alg:equalizeBranches}
\end{algorithm*}

\begin{algorithm*}      
    \SetAlgoLined
    \DontPrintSemicolon
    
    \SetKwProg{alignCFG}{Procedure}{}{}
    \SetKwProg{alignNodes}{Procedure}{}{}
    
    \SetKwFunction{alignCFGFNName}{AlignCFG}
    \SetKwFunction{alignNodesFNName}{AlignNodes}
    \SetKwFunction{computeDistanceFromNode}{ComputeDistanceFromNode}
    
    \alignCFG{\alignCFGFNName{g: CFG, v: Node}}{
    subgraph $\leftarrow$ \extractSubgraphFNName{g, v} \\
    \tcp{Compute distances from $v$ to all nodes in $subgraph$}
    pathLengths $\leftarrow$ \computeDistanceFromNode{subgraph, v} \\
    levels $\leftarrow$ \{ l $\; \vert \; u \in$ \nodes{subgraph} $\land$ pathLengths[u] = l \} \\
    \ForAll{l $\in$ levels}{
        levelNodes $\leftarrow$ \{ $u \; \vert \; u \in$ \nodes{subgraph} $\land$ pathLengths[u] = l \} \\
        \alignNodesFNName{subraph, levelNodes} \\
    }
    }
    
    \SetKwFunction{countInstruction}{CountInstructions}
    \SetKwFunction{selectReferenceNode}{SelectReferenceNode}
    \SetKwFunction{getNodeInstruction}{GetNodeInstruction}
    \SetKwFunction{latency}{Latency}
    \SetKwFunction{isBranch}{IsBranch}
    \SetKwFunction{isReturn}{IsReturn}
    
    \SetKwFunction{selectRegister}{SelectRegister}
    \SetKwFunction{getNOPInstruction}{GetNOPInstruction}
    \SetKwFunction{getBranchInstruction}{GetBranchInstruction}
    \SetKwFunction{insertInstruction}{insertInstruction}
    
    \alignNodes{\alignNodesFNName{g: CFG, ns : NodeSet}}{
    index $\leftarrow$ 0 \\ 
    \While{True}{
        nodeLengths $\leftarrow$ \{ node: \countInstruction{node} $\; \vert \;$ node $\in$ \nodes{g} \} \\
        candidates $\leftarrow$ \{n $\; \vert \;  n \in$ \nodes{g} $\land$  nodeLengths[n] = \max{nodeLengths} \} \\
        referenceNode $\leftarrow$ \selectReferenceNode{candidates} \\
        referenceInstruction $\leftarrow$ \getNodeInstruction{referenceNode, index} \\
        
        \ForAll{node $\in$ \{n $\vert$ n $\in$ \nodes{g} $\land$ n $\neq$ referenceNode \}}{
            \uIf{index $<$ nodeLength[node] $\; \land \;$  \latency{\getNodeInstruction{node, index}} = \latency{referenceInstruction}}{
                continue
            }
            \uIf{\isBranch{referenceInstruction}}{
                newInstruction $\leftarrow$ \getBranchInstruction{}\\ 
                \insertInstruction{node, newInstruction} \\
            } \uElse{
                reg $\leftarrow$ \selectRegister{} \\ 
                newInstruction $\leftarrow$ \getNOPInstruction{\latency{referenceInstruction}, reg} \\ 
                \insertInstruction{node, newInstruction} \\
            }
        }    
    }
    }
    
    \SetKwProg{selectReferenceNode}{Function}{}{}
    \SetKwFunction{selectReferenceNodeFNName}{SelectReferenceNode}

    \selectReferenceNode{\selectReferenceNodeFNName{candidates: NodeSet, index: Integer}}{
        \For{n $\in$ candidates}{
            candidateInstruction $\leftarrow$ \getNodeInstruction{n, index} \\
            \uIf{$\lnot$ (\isBranch{candidateInstruction} $\lor$ \isReturn{candidateInstruction}) }{
                \KwRet n \\
            }
        }
        \KwRet candidates[0] \\
    }
    
       
    \caption{Align CFG}
    \label{alg:align}
\end{algorithm*}

\begin{algorithm*}
    \SetAlgoLined
    \DontPrintSemicolon
    
    
    \SetKwProg{closeTimingLeaks}{Procedure}{}{}
    \SetKwProg{alignNodes}{Procedure}{}{}
    
    \SetKwFunction{closeTimingLeaksName}{CloseTimingLeaks}
    \SetKwFunction{alignNodesFNName}{AlignNodes}
    \SetKwFunction{computeDistanceFromNode}{ComputeDistanceFromNode}
	\SetKwFunction{removeCycles}{RemoveCycles}    
    \SetKwFunction{restoreCycles}{RestoreCycles}    
    
    \closeTimingLeaks{\closeTimingLeaksName{g: CFG, instrs: [Instruction]}}{	
     	targetNodes $\leftarrow$ $\{ n \vert n \in $\nodes{g}$ \land  \exists instr \in instrs : instr \in n \}$ \\
     	removedEdges $\leftarrow$ \removeCycles{g} \\
     	\ForAll{node $\in$ targetNodes}{
     		\equalizePathsFnName{g, node} \\
     		\equalizeBranchesFNName{g, node} \\
     	}

     	\ForAll{node $\in$ targetNodes}{
     		\alignCFGFNName{g, node} \\	
     	}
		
		\restoreCycles{g, removedEdges} \\
     	
    }
    \caption{Close timing leaks}
    \label{alg:closeleaks}
\end{algorithm*}

\chapter{Implementation}

\chapter{Evaluation}
\section{Benchmark Suite}
Winderix et. al. have created the first benchmark suite of programs with timing side-channel vulnerabilities. This suite consists of a 
collection of synthetic programs with a wide range of control-flow patterns as well as third party benchmark programs from different sources \cite{WinderixHans}. 
To evaluate the proposed algorithm a subset of this benchmark suite was selected. 

All programs that contain loops inside vulnerable branches were discarded from the synthetic programs in the benchmark suite, 
since they are not supported by the proposed algorithm. 
The original authors of the Nemesis attack provide two case studies to demonstrate their attack. 
The first case study is a password comparison routine from the Texas Instruments MSP430 Bootstrap Loader (BSL). 
The second case study is secure keypad application that guarantees secrecy of its PIN code \cite{Nemesis}.
Both of these are included in the benchmark suite created by Winderix et. al and are also selected as a benchmark for the proposed algorithm. 

The implementation of Nemesis and the benchmark suite created by Winderix et. al are implemented for the Sancus environment. Any pieces of code specific to this
environment have been removed from the benchmark programs. The semantics of the programs remain unchanged. 

One additional synthetic programs was added to the benchmark suite to evaluate a case that was not yet covered. This program contains a call to a function that 
modifies a non-local variable though a pointer. This function is only called in one branch of a secret-dependent branch. 

\section{Experiment Setup}
The algorithm is evaluated using three metrics. The first metric aims to measure the effectiveness of the algorithm. A static analysis tool was developed to verify 
for a given program whether or not the program satisfies the Nemesis-sensitive property as specified in section \ref{sec:property}.
Given a program and a set of secret-dependent branches, this tool partitions instructions into sets according to their positions in secret dependent branches. Following the notation of section \ref{sec:property}, let \textit{ep} be a secret dependent branch, and let $ep^n$ be the n'th instruction in a region, then define the set 
\begin{equation} \label{eq:toolSets}
    ep_i = \{ ep^n |i = n \land  (ep^n \in region_{then}(ep) \lor ep^n \in region_{else}(ep)\}
\end{equation}
The static analysis verifies that both the regions have the same number of execution points, and that for each set $ep_i$ it holds that all 
instruction have the same latency.

The second metric aims to measure the correctness of the algorithm. The algorithm is considered to work correctly if it does not change the program output. For each program in the benchmark suite a number of input values were determined such that all possible paths of the program control flow were covered.
These values were supplied as inputs to both the original program and the balanced program, generating two output values. The output values were then compared to 
verify that the algorithm correctly modified the program without changing the output. 

The effect on the program's performance is evaluated by measuring the increase in the sum of the latencies along paths in the programs CFG. 
To measure this increase CFG are constructed from the original binary and from the modified binary. 
For each path in the original CFG its corresponding path in the modified CFG is determined. TO do so a mapping is created that maps all nodes in the original CFG to their corresponding node in the balanced CFG. 
This mapping takes into account the condition of a branching instruction, and can be defined inductively. 
The root of the original CFG is mapped to the root of the root of the balanced CFG. 
If two nodes are mapped and they both have one successor then their successors are mapped. 
If two nodes are mapped and they have two successors, then then nodes that are reached if the branching condition is true are mapped,
and those that are reached if the condition is false are mapped.

Formally, let $G$ denote the original CFG, and let $G'$ denote the modified CFG. Let $succ(n)$ be the successors of node $n$, and let $succT(n)$ be 
the successor of node N when the branching condition is true. Let $F$ be the function that maps between the two CFGs.  
\begin{enumerate}
    \item $\begin{aligned}[t]
    F(root(G)) = root(G')
\end{aligned}$
\item $\begin{aligned}[t]
    F(n) = n' \land succ(n) = \{s\} \land succ(n')=\{s'\} \\ 
    \implies F(s)=s'
\end{aligned}$
\item $\begin{aligned}[t]
    F(n) = n' \land succ(n) = \{s, t\} \land succ(n')=\{s', t'\} \land succ_T(n)=s \land succ_T(n') = s'\\
    \implies F(s)=s', F(t) = t'
\end{aligned}$
\end{enumerate}
Let $p$ be a path in $G$
$$ p: p_1 \rightarrow p_2 \rightarrow ... \rightarrow p_n$$
Then its corresonding path in $G'$ is defined as follows 
$$ p': F(p_1) \rightarrow F(p_2) \rightarrow ... \rightarrow F(p_n)$$

This definition requires that $G$ and $G'$ are isomorphic. If during the first stage of the algorithm additional nodes were inserted in the CFG
then this will not be true. Therefore before being able to evaluate the effect on runtime the first stage of the algorithm has to be reapplied on $G$ such 
that it is isomorphic to $G'$

To evaluate the effect on runtime performance the sum of the latencies along all relevants paths in G are compared to the sum of the latencies of their corresponding paths. A relevant path is a path that starts in secret-depdendent node and ends in a final node of one of the branches. Any nodes that do not belong to such a path 
are not affected by the algorithm and are therefore not considered in this evaluation. 

    
\section{Discussion}

The results of the experiments are summarized in figure \ref{fig:experiment results}. 
The results show that  the algorithm was able to ensure the Nemesis sensitive property holds for all programs, as verified by the static analysis tool described in the previous section. 

In all but one test case the algorithm had no effect on the program output. 
The erroneous test case contains a call to a function that modifies the global state of the program in one of its secret dependent branches. 
During balancing of the program this function call is copied to the other branch. 
Because the function call has side effects the final output of the program is different. 


The effect on performance ... 

\begin{figure}
    \centering
    \begin{tabular}{ l | c | c | c c c c c c }
    Name & Effectiveness & Correct & \multicolumn{6}{c}{Performance} \\ 
     & & & path1 & path2 & path3 & path4 & path5 & path6\\
     \hline 

    call        & Y & Y & 1.32 & 1.17 & &  & & \\  
    call2       & Y & N & 1.25 & 1.22 & &  & & \\
    diamond     & Y & Y & 1.35 & 1.06 & 1.06 & & &  \\ 
    fork        & Y & Y & 1.43 & 1.15 & & & &  \\  
    ifcompound  & Y & Y & 1.31 & 1.26 & 1.11 & 1.11 & 1.09 & 1.09  \\
    indirect    & Y & Y & 1.44 & 1.32 & 1.20 & 1.11 & & \\ 

    multifork   & Y & Y & 1.80 & 1.58 & 1.41 & 1.41 & &   \\
    triangle    & Y & Y & 1.30 & 1.16 & & & &  \\
  	\hline
  	bsl         & Y & Y & 1.42 & 1.00 & & & &  \\ 
	keypad      & Y & Y & 1.67 & 1.55 & 1.45 & 1.02 & 1.12 &   \\  
%    sharevalue  & Y & Y & 1.25 & 1.0 & & & &  \\
    \end{tabular}
    \caption{experiment results. Increase in performance is expressed as a percentage increase of the sum of the latencies along the path}
    \label{fig:experiment results}
\end{figure}


\chapter{Related Work}
\label{cha:related}
This chapter discusses a number of countermeasures that have been proposed for closing timing side-channels. 
These can broadly be categorized as being either software-based or hardware-based. 
Hardware-based solution are based on modification to the architecture and are discussed in section \ref{sec:hardware}, whereas the software-based solutions discussed in section \ref{sec:software} are implemented at the language level \cite{Barthe}.  
\section{Software-based approaches}
\label{sec:software}

Popular software-based approaches to closing timing-leaks include constant-time policies and the program counter model \cite{Barthe}. 
Constant time policies require that memory access and control-flow should not depend on secret data. 
Unfortunately writing code that adheres to these policies can be difficult, since it requires knowledge of the compiler, and requires developers to deviate from 
conventional programming practices.
A number of solutions have been proposed to verify if a program adheres to constant-time policies \cite{verify-constant-time, Barthe}.

Molnar et al. \cite{programcounter} first introduced the program counter model in their work, proposing methods for the detection and mitigation of control-flow side channel attacks. 
The authors consider the case where an adversary is able to make an observation of a side channel at each step of the computation. 
The result is a sequence of observations $T= (T_1, T_2, ... T_n)$ called a \textit{transcript}.
The \textit{program counter model} or \textit{PC model} is then the model where each value of the transcript is the processor's program counter during the computation. 
A program is then said to be  PC-secure if this transcript is secure. 
The authors state that \textit{any program that is PC-secure will also be secure against timing attacks}.
Based on this definition of PC-security the authors introduce a code transformation for creating PC-secure C code.
The authors note that they made a number of assumptions in their work. 
Although these assumptions do not hold for a number of architectures the authors are confident 
that they are reasonable for some embedded devices. 	

Winderix et al. \cite{WinderixHans} recently proposed a new algorithm that aligns instructions in corresponding branches in a way similar to the algorithm outlined in this text, 
by first equalizing path lengths and then aligning nodes. 
They have implemented their algorithm for the Sancus architecture as a compiler pass in the LLVM structure. Their solution supports loops nested within secret-dependent branches 
and as a result covers a larger set of programs. Unlike the solution proposed in this work, however, their solution cannot be applied to off-the-shelf binaries, as it requires access to the source code and recompilation.  
 
\section{Hardware-based approaches}
\label{sec:hardware}

An orthogonal approach is to close timing leaks using hardware-based solutions. 
Recently Busi et al. \cite{busi} proposed an approach to extend architectures with non-interruptible enclaved execution such that they can also support interruptions without breaking 
the existing isolation properties. Based on this approach, they proposed  a design for interruptible enclaves that are resistent against interrupt attacks. 
This design is based on an earlier version of Sancus with non-interruptible enclaves. They modify the architecture to add padding cycles whenever the enclave is interrupted, effectively closing timing leaks. 

A limitation of their approach is that their design is based on the assumption that the timing of instruction is predictable. This is generally not the case for more complex architectures such as Intel's x86\_64.
Additionally their approach requires hardware modifications, which means it cannot be applied to off-the-shelf and existing devices. 



\chapter{Conclusion}

\section{Discussion}
One limitation of the proposed algorithm is the lack of support for cycles within secret-dependent branches.
However, the evaluation shows that the algorithm is effective in closing timing leaks in programs where this does not occur. 
This indicates that it is possible modify the algorithm such that it is able to close these leaks even in the presence of cycles. 

The root cause of the issue is the fact that a section of the first branch will be executed a higher number of times than the corresponding section in the second branch. 
As a result one of the latency traces will be longer, even when the relevant nodes are aligned.
A solution that addresses this would have to make more extensive changes to the program. 
Before aligning the nodes, the structure of the cycle would have to be duplicated into the second branch such that the corresponding section is is executed the same number of times. 
This requires additional analysis to determine which register contains the loop counter and how many time it is incremented. 
The duplication of the cycle also requires more significant changes than those implemented by the current algorithm. % hier nog aangeven waarom / welke aanpassingen? 
Due to the added complexity such a solution is not included in the proposed algorithm and is left to future research. 

A second limitation is the incomplete coverage of the latency data. There are certain instructions for which there is no available data. 
If these instructions are encountered in branches of  a secret-dependent branching instruction then the algorithm is not able to close the timing leaks. 
However this issue was not encountered during evaluation of the algorithm. 
This indicates that the most commonly used instructions are present in the data and that the coverage of the data is sufficient to close timing leaks in most programs. 

The evaluation of the effectiveness of the algorithm is based on a statistical analysis of the program before and after alignment. 
If the instruction latencies are fully deterministic then the static analysis tool can correctly predict the actual run-time instruction latencies. The resulting analysis 
is then sufficient for demonstrating that the algorithm correctly closes all timing leaks. 
In the presence of advanced micro-architectural features the instruction latencies are to some extend random, however. 
As a result the run-time latencies diverge from the predicted latencies. 
Although the results indicate that all timing leaks are closed it is possible that some differences still exist between branches because of these random variations. 
Because of this additional experiments are needed to fully verify if the algorithm is effective for complex architectures with non-deterministic latencies.
Empirical measurements in the form of latency traces are needed to determine if an attacker can still distinguish between branches of a secret-dependent conditional node. 


\section{Future work}


\backmatter 

\bibliographystyle{abbrv}
\bibliography{references}

\end{document}
