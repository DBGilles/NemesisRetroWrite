
\chapter{Conclusion}

\section{Discussion}
One limitation of the proposed algorithm is the lack of support for cycles within secret-dependent branches.
However, the results show that the algorithm is effective in closing timing leaks. 
This indicates that it is possible modify the algorithm such that it is able to close these leaks even in the presence of cycles. 

The root cause of the issue the fact that a section of the first branch will be executed a higher number of times than the corresponding section in the second branch. 
As a consequence one of the latency traces will be, even when the relevant sections are aligned.
A solution that addresses this would have to make more extensive changes to the program. 
Before aligning the nodes, the structure of the cycle would have to be duplicated into the second branch such that the corresponding section is is executed the same number of times. 
This requires additional analysis to determine which register contains the loop counter and how many time it is incremented. 
The duplication of the cycle also requires more significant changes than those implemented by the current algorithm. % hier nog aangeven waarom / welke aanpassingen? 
Due to the added complexity such a solution is not included in the proposed algorithm and is left to future research. 

A second limitation is the incomplete coverage of the latency data. There are certain instructions for which there is no available latency data. 
If these instructions are encountered in branches of  a secret-dependent branching instruction then the algorithm is not able to close the timing leaks. 
However this issue was not encountered during evaluation of the algorithm. 
This indicates that the most commonly used instructions are present in the data and that the coverage of the data is sufficient to close timing leaks in most programs. 

\section{Future work}