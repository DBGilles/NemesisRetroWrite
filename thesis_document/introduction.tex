
\lstdefinelanguage
   [x64]{Assembler}     % add a "x64" dialect of Assembler
   [x86masm]{Assembler} % based on the "x86masm" dialect
   % with these extra keywords:
   {morekeywords={CDQE,CQO,CMPSQ,CMPXCHG16B,JRCXZ,LODSQ,MOVSXD, %
                  POPFQ,PUSHFQ,SCASQ,STOSQ,IRETQ,RDTSCP,SWAPGS, %
                  rax,rdx,rcx,rbx,rsi,rdi,rsp,rbp, %
                  r8,r8d,r8w,r8b,r9,r9d,r9w,r9b, %
                  r10,r10d,r10w,r10b,r11,r11d,r11w,r11b, %
                  r12,r12d,r12w,r12b,r13,r13d,r13w,r13b, %
                  r14,r14d,r14w,r14b,r15,r15d,r15w,r15b}} % etc.


\chapter{Introduction}
\label{cha:introduction}
As a result of the increasing popularity of IoT devices and cloud-computing services, software is often executed on third party platforms. 
Embedded devices often support software extensibility, allowing users to install additional software onto their device. 
Large cloud-service providers such as Amazon and Microsoft allow users to run virtually any piece of software on their devices. 
To ensure the various components on these systems are isolated from each other a sizeable software layer is often introduced in the form of an operating system or 
a hyper visor.
Unfortunately, however, this software layer is very difficult to get sufficiently secure \cite{psma}.
As a result it has become increasingly important to protect software from attacks even in the presence of a compromised system. 
One line of research is the use of Protected Module Architecures (PMAs). 
These are architectures that enforce isolation of components directly in hardware.
PMAs ensure that the untrusted OS cannot access the data or code of a protected module \cite{Nemesis}. 

Recent research has shown, however, that it is still possible to extract secret data from protected modules through the use of controlled-channel attacks. 
Controlled-channel attacks are a type of side-channel attack that leverage an attackers control of the system to extract more information \cite{Xu}.	
Side-channel attacks aim at extracting secrets from a system through measurement of physical parameters \cite{side-channel}. 
In the case of controlled-attacks an attacker has increased capabilities for extracting data, since he can use system events such as as page faults, scheduling decisions, and interrupts to open additional side-channels. 
Research has shown that these attacks are able to extract information even from modules that are protected by PMAs. 

Recently Van Bulck et al. \cite{Nemesis} have been able to exploit the CPU interrupt mechanism to leak micro-architectural instruction timings from protected modules. 
Their attack, called Nemesis, exploits the property that all arriving interrupts are only served after the current instruction is done executing. 
As a result, the interrupt latency, the delay between arrival of the interrupt and the execution of the first instruction in the interrupt service routine, increases with the number of cycles left to execute.
An attacker with control over the system can exploit this by carefully timing interrupts and measuring the interrupt latency to infer the duration of the interrupted instruction. 

Van Bulck et al. \cite{Nemesis} convincingly demonstrate that it is possible to use these Nemesis-type interrupt attacks to leak information about secret-dependent control flow of the program. 
The authors require two branches of a conditional jump that contain instruction with different execution times. 
If the control flow depends on a secret the attacker is then able to infer some information about it. 

A number of countermeasures have been proposed for closing timing side-channels, both software-based approaches and hardware-based approaches. 
Hardware based-approaches aim to close side-channels by modifying the architecture. 
A limitation of hardware-based approaches is that they require the replacement or modification of existing devices. This makes these approaches difficult
to apply in the field. 
Software-based approaches are implemented at the language level. A number of transformations have been proposed that can automatically close timing leaks. 
Additionally tools exists that can verify if a program is safe from leaks. Unfortunately these solutions often require recompilation of the program. This means
that the source code of the program has to be available. As a result these solutions are not generally applicable to commercial off-the-shelf binaries. 

The solution proposed in this thesis makes use of binary rewriting to circumvent this issue. Binary rewriting is the alteration and transformation of a compiled program without having the source code at hand \cite{rewriting-survey}. 
The target binary is decompiled and reconstructed into an assembly file and additional dummy instructions are inserted into the program to close any timing leaks. 
After applying all transformations the assembly file can then be compiled into an executable using any existing compiler. 
Unlike previous solutions this algorithm can be applied to commercial off-the-shelf binaries. 

\section{Thesis Goal and Outline}
This paper presents a novel algorithm for automatically transforming a program in order to remove any timing leaks. It achieves this by addressing the core cause of the vulnerability: differences in 
latencies between corresponding instructions in secret-dependent branches. Corresponding instructions are instructions that are the same distance away from a branching instruction. 
The proposed algorithm inserts additional instructions such that all corresponding instructions have the same latency without changing affecting the program output. 
Unlike previous solutions that require recompilation, the proposed algorithm transforms the target program through binary rewriting. 

The main contributions of this paper are:
\begin{enumerate}
\item The paper presents a novel algorithm for automatically transforming a program to remove any timing leaks. Unlike previous solutions it is applicable to off-the-shelf binaries. 
\item The paper presents an implementation of this algorithm for the Intel x86-64 architecture. 
\item The paper presents an evaluation of the algorithm based on a suite of benchmark programs. 
\end{enumerate}

Chapter \ref{cha:background} will provide additional background information on PMA's, the Nemesis attack, and binary rewriting. 
Chapter \ref{cha:design} outlines the design of the proposed algorithm, and formalizes the property it aims to enforce.  
Aspects specific to the implementation of the algorithm are further discussed in chapter \ref{cha:implementation}. 
A number of experiments have been designed and run to evaluate if the proposed algorithm is effective and correct. 
Chapter \ref{cha:evaluation} describes the setup of these experiments and discusses the results. 
Next a number of related works is discussed in chapter \ref{cha:evaluation}.
Finally chapter \ref{cha:conclusion} provides some discussions about the benefits and limitations of the proposed algorithm and suggests 
future work. 